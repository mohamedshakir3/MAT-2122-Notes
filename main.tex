\documentclass[openany]{report}
\usepackage[utf8]{inputenc}

\usepackage{stylesheets}
\usepackage{lecture_notes_styles}
\title{MAT 2125 Lecture Notes}
\author{Last Updated:}

\begin{document}

\maketitle

\tableofcontents

\chapter{The Real Numbers $\real$}
TBC.
\chapter{Completeness of $\real$, Absolute Value, Sequences }
TBC.
\chapter{Convergence of Sequences}
TBC. 
\chapter{Properties of Convergence, Squeeze Theorem, Monotone Sequences}
TBC. 
\chapter{Subsequences, Cauchy Sequences}
TBC.
\chapter{Limsup and Liminf}
TBC.

%----------------------------Lecture 7 ----------------------------------
\chapter{Series}
\textbf{Recall:}
\[\infser{a} = \lim_{N\rightarrow \infty} \sum_{n=1}^N a_n\]
$\infser{a}$ "diverges" if above limit does not exisit.
\begin{prop}
    Suppose $\infseq{a}, \infseq{b}$ converges. Then
    \begin{enumerate}[label=(\roman*)]
        \item 
        \[\sum_{n=1}^\infty a_n + b_n = \infser{a} + \infser{b}\]
        \item
        \[\sum_{n=1}^\infty cb_n = c\infser{b} \ \forall c \in \real\]
    \end{enumerate}
    This says 
    \[V \coloneqq \{\infseq{a}: \infser{a} \ converges\}\]
    is a vector space over $\real$.
\end{prop}

\begin{center}
    \textbf{Note:}
    \[\left(\sum_{n=1}^N a_n\right) \left(\sum_{n=1}^N b_n\right) \neq \sum_{n=1}^\infty a_nb_n\]
\end{center}

\begin{proof}
    \textbf{Exercise.}
\end{proof}

\begin{prop}
    \begin{center}
        $\infser{a}$ converges $iff$ $\sum_{n=k}^\infty a_k$ converges.
    \end{center}
\end{prop}
\begin{proof}
    \textbf{Exercise.}
\end{proof}

\textbf{Example:} TBC.

\begin{prop}
    If $\infser{a}$ converges, then $a_n \rightarrow 0$.
\end{prop}




\section{Divergence Test}

\begin{prop}[Divergence Test]
    If $a_n\not\rightarrow 0$, then $\infser{a}$ diverges.
\end{prop}
\begin{proof}
    $\sum_{n=1}^\infty a_n$ converges $\implies S_n \rightarrow L$ for some $L$, where 
    \[S_n \coloneqq \sum_{n=1}^N a_n\]
    So, 
    \[a_n = S_n - S_{n-1}\]
    By the Algebra of Limits, 
    \begin{align*}
        \lim_{n\rightarrow \infty} a_n &= \lim_{n\rightarrow \infty} S_n - \lim_{n\rightarrow \infty} S_{n-1}\\
        &= L - L = 0
    \end{align*}
\end{proof}
\textbf{Example:} TBC.

\section{Convergence Tests}
\begin{prop}[Boundedness Test]
    Let $\infseq{a}$ bea  sequence, if 
    \begin{enumerate}[label=(\roman*)]
        \item $a_n \geq 0$
        \item There is an upper bound on the parital sums
        \[\exists M > 0 \ s.t \ \sum_{n=1}^N a_n \leq M\]
    \end{enumerate}
    Then $\infser{a}$ converges.
\end{prop}
\begin{proof}
    Let 
    \[S_N \coloneqq \sum_{n=1}^N a_n\]
    Then 
    \begin{align*}
        S_{N+1} &= S_N + a_{N+1}\\ 
        &\geq S_n
    \end{align*}
    So by the Monotone Convergence Criterion, $(S_N)_{N=1}^\infty$ converges $\iff$ it is bounded avove. By ($ii$), it is bounded.
\end{proof}

\begin{prop}[Comparison Test]
    
\end{prop}

%----------------------------Lecture 8 ----------------------------------
\chapter{Ratio, Root, Alternating Series, and Integral Test, Cauchy Convergence, Topology of $\real^d$}
\section{Ratio Test}
\begin{prop}[Ratio Test]
    Let $\infseq{a}$ be a sequence of nonzero elements.
    \begin{enumerate}[label=(\roman*)]
        \item If 
        \[\limsup_{n \rightarrow \infty} \left| \frac{a_{n+1}}{a_n} \right|< 1\]
        then $\infser{a}$ converges absolutely.
        \item If 
        \[\liminf_{n\rightarrow \infty} \left| \frac{a_{n+1}}{a_n} > 1 \right| > 1\]
        then $\infser{a}$ diverges.
    \end{enumerate}
\end{prop}

\begin{proof}
    ~\newline
    \begin{enumerate}[label=(\roman*)]
        \item Let 
        \[q = \limsup_{n\rightarrow \infty} \left| \frac{a_{n+1}}{a_n} \right|\]
        where $q < 1$, $\exists r \in (q,1)$. By the definition of $\limsup$, 
        \[\left| \frac{a_{n+1}}{a_n} \right| \leq r \ \forall n \geq n_0\]
        for some $n_0 \in \nat$.
        \begin{align*}
            \left| \frac{a_{n_0+1}}{a_{n_0}} \right| &\leq r\\
            \left| a_{n_0+1} \right| &\leq r|a_{n_0}|\\
            \left| a_{n_0+2} \right| &\leq r|a_{n_0+1}| \leq r^2|a_{n_0}|\\
        \end{align*}
        By induction, we have
        \[0 \leq |a_{n_0 + k} \leq r^k |a_{n_0}|\]
        By the comparison test, 
        \[\sum_{k=1}^\infty |a_{n_0+ k}|\]
        converges since 
        \[\sum_{k=1}^\infty r^k|a_{n_0}|\]
        is a geometric sequence and $0 \leq r < 1$. 
        \[\sum_{n=1}^\infty |a_n|\]
        Converges, thus $\infser{a}$ converges absolutely.
        \item Let 
        \[q = \liminf_{n \rightarrow \infty} \left| \frac{a_{n+1}}{a_n}\right| > 1\]
        so $\exists n_0$ such that 
        \[\left| \frac{a_{n+1}}{a_{n}} \right| > 1\]
        for all $n \geq n_0$. Then
        \[|a_{n_0 + k} \geq |a_{n_0}| \ \forall k \geq 0 \]
        So $a_n \not\rightarrow 0$ as $n \rightarrow 0$, thus by the divergence test. $\infser{a}$ diverges.
    \end{enumerate}
\end{proof}

\begin{center}
    \red{\textbf{Note: The ratio test does not tell us anything when the limit is 1.}}
\end{center}
\textbf{For example:} 
\[\sum_{n\geq 1} \frac{1}{n}\]
diverges, but 
\[\frac{\frac{1}{n+1}}{\frac{1}{n}} = \frac{n}{n+1} \rightarrow 1\]
But on the other hand, 
\[\sum_{n\geq1} \frac{1}{n+1}\]
converges, and 
\[\frac{\frac{1}{n+1(n+2)}}{\frac{1}{n(n+1)}} = \frac{n+1}{n+2} \rightarrow 1\]
\section{Root Test}
\begin{prop}[Root Test]
    Let $\infseq{a}$ be a sequence of real numbers.
    \begin{enumerate}[label=(\roman*)]
        \item If 
        \[\limsup_{n\rightarrow \infty} \sqrt[n]{|a_n|} < 1\]
        then $\infser{a}$ converges absolutely.
        \item If 
        \[\limsup_{n\rightarrow \infty} \sqrt[n]{|a_n|} > 1\]
        then $\infser{a}$ diverges.
    \end{enumerate}
\end{prop}

\begin{proof}
    ~\newline
    \begin{enumerate}[label=(\roman*)]
        \item Let 
        \[q = \limsup_{n\rightarrow \infty} \sqrt[n]{|a_n|} < 1\]
        Then there exists $r \in (q,1)$ such that $\exists n_0$ 
        \[\sqrt[n]{|a_n|} \leq r\]
        for all $n \geq n_0$. Then,
        \[0 \leq |a_n| \leq r^n\]
        Therefore $\sum_{n=n_0}^\infty r^n$ converges since $0 < r < 1$. Then by the comparison test, $\sum_{n=n_0}^\infty |a_n|$ converges.
        \item Let
        \[q = \limsup_{n\rightarrow \infty} \sqrt[n]{|a_n|} > 1\]
        From exercise 2.7.3, there are infinitely many 
        \[\sqrt[n]{|a_n|} \geq 1 \implies |a_n| \geq 1\]
        Thus $a_n \not\rightarrow 0$ as $n \rightarrow \infty$, thus by the divergence test, $\infser{a}$ diverges.
    \end{enumerate}
\end{proof}

\textbf{Examples:}
\begin{itemize}
    \item 
    \[\sum_{n=1}^\infty \left(\frac{1}{2}\right)^n\]
    \[\sqrt[n]{\left(\frac{1}{2}\right)^2} = \frac{1}{2}\]
    \item
    \[\infser{\left(\frac{n}{2n+1}\right)}\]
    \[\sqrt[n]{\left(\frac{n}{2n+1}\right)^n} = \frac{1}{2^{n+1}} \rightarrow \frac{1}{2}\]
    \item 
    \[\infser{\frac{1}{n}}\]
    \[\sqrt[n]{\frac{1}{n}} = \frac{1}{n^{\frac{1}{n}}} = \frac{1}{e^{\frac{\ln n}{n}}} \rightarrow 0\]
\end{itemize}
\section{Alternating Series Test}
\begin{prop}[Alternating Series Test]
    Let $\infseq{a}$ be a sequence of real numbers. Suppose
    \begin{enumerate}[label=(\roman*)]
        \item $\infseq{a}$ is decreasing
        \item $\lim_{n\rightarrow \infty} a_n = 0$
    \end{enumerate}
    Then 
    \[\sum_{n=1}^\infty (-1)^{n+1}a_n\]
    converges. Moreover, for any $N$, 
    \[\sum_{n=1}^{2N} (-1)^{n+1}a_n \leq \sum_{n=1}^\infty (-1)^{n+1}a_n \leq \sum_{n=1}^{2N-1} (-1)^{n+1}a_n\]
\end{prop}
\begin{proof}
    $a_n \geq 0$ $\forall n$, since $a_n \rightarrow 0$ as $n \rightarrow \infty$, and $\infseq{a}$ is decreasing. Let
    \[S_N \coloneqq \sum_{n=1}^N (-1)^{n+1}a_n\]
    If $N$ is even, 
    \[S_{N+2} = S_N + a_{N+1} - a_{N+2} \geq S_N\]
    Since 
    \[a_{N+2} \leq a_{N+1}\]
    So $(S_{2N})_{N=1}^\infty$ is an increasing sequence and $(S_{2N -1})_{N=1}^\infty$ is a decreasing sequence. So by the monotone convergence criterion, both sequence converge. 
    \[S_{2N - 1} + a_{2N} = S_{2N}\]
    \[a_{2N} \xrightarrow{N \rightarrow \infty} 0 \]
    So, 
    \[\lim_{n\rightarrow \infty} S_{2N-1} = \lim_{N \rightarrow \infty} S_{2N} = L \implies \lim_{N\rightarrow \infty} S_N = L\]
    $(S_{2N})^\infty_{N=1}$ is increasing, so 
    \[L = \sup \{(S_{2N})_{N=1}^\infty\} \implies S_{2N} \leq L\]
    Similarily, $(S_{2N-1})^\infty_{N=1}$ is decreasing, so
    \[L = \inf\{(S_{2N_1})_{n=1}^\infty\} \implies S_{2N - 1} \geq L\] 

\end{proof}

\section{Integral Test}

\begin{prop}
    Let $f: [1,\infty) \rightarrow \real$. Suppose that
    \begin{enumerate}[label=(\roman*)]
        \item $f(x) \geq 0$ $\forall x \in [1,\infty)$
        \item $f$ is decreasing
    \end{enumerate}
    Then 
    \begin{center}
        $\sum_{n=1}^\infty f(n)$ converges $\iff$ the improper integral $\int_1^\infty f(x) \ dx$ converges.
    \end{center}
\end{prop}

\section{Cauchy Convergence Criterion for Series}

\begin{prop}
    Let $\infseq{a}$ be a sequence of real numbers. Then
    \begin{center}
        $\sum_{n=1}^\infty a_n$ converges $\iff$ $\forall \epsilon > 0$ $\exists N_0$ such that $N \leq M \leq N_0$, $|\sum_{n=M}^N a_n| \epsilon$. 
    \end{center}
\end{prop}
\begin{proof}
    Let 
    \[S_N \coloneqq \sum_{n=1}^N a_n\]
    Then $\sum_{n=1}^\infty a_n$ converges $\iff$ $(S_N)_{N=1}^\infty$ converges. Cauchy convergence criterion for sequences says that 
    \begin{center}
        $(S_N)^\infty_{n=1}$ converges $\iff$ it is cauchy.
    \end{center}
    \[\iff \forall \epsilon > 0 \ \exists N_0 \ s,t \ |S_N - S_M| < \epsilon \ \forall N,M \geq N_0\]
\end{proof}

\section{Topology of $\real^d$}
\subsection{Norms}
\begin{definition}
    A norm on $\real^d$ is a function $||\cdot||: \real^d \rightarrow [0,\infty)$ satisfying the following properties:
    \begin{enumerate}[label=(\roman*)]
        \item $||a|| = 0 \iff a = (0,\dots,0)$
        \item $||ca|| = |c|\cdot||a|| \ \forall c \in \real, a \in \real^d$
        \item $||a+b|| \leq ||a|| + ||b|| \ \forall a,b \in \real^d$
    \end{enumerate}
\end{definition}
The euclidean norm of $\real^d$ is given by
\[||a,\ldots,a_d||_2 = \sqrt{\sum_{i=1}^d a_i^2}\]
The dot prdouct on $\real^d$ is given by
\[(a_1,\ldots,a_n) \cdot (b_1,\ldots,b_n) = \sum_{i=1}^d a_ib_i\]
We also have the $l_1$-norm
\[||(a_1,\ldots, a_n)||_1 =\sum_{i=1}^d |a_i|\]
And the $l_\infty$-norm
\[||(a_1,\ldots, a_d)||_\infty = \max\{|a_1|,\ldots,|a_d|\}\]
\begin{prop}
    Let $a,b\in\real^d$, write $||\cdot||$ for the euclidean norm on $\real^d$. Then
    \begin{enumerate}[label=(\roman*)]
        \item \textbf{Cauchy Schwarz Inequality:} 
        \[|a\cdot b| \leq ||a||\cdot||b||\]
        \item \textbf{Triangle Inequality:}
        \[||a+b|| \leq ||a|| + ||b||\]
        \item $||\cdot||$ is a norm on $\real^d$
    \end{enumerate}
\end{prop}
\begin{proof}
    ~\newline
    \begin{enumerate}[label=(\roman*)]
        \item Consider the equadratic function
        \begin{align*}
            P(t) &= ||a+tb||^2\\
                &=(a+tb)\cdot(a+tb)\\
                &=a\cdot a + 2a\cdot tb + tb\cdot tb\\
                &=||a||^2 + 2t(a\cdot b) + t^2||b||^2
        \end{align*}
        The discriminant of $P(t)$ is less than or equal to 0,
        \begin{align*}
            (2a\cdot b)^2 - 4||a||^2||b||^2 &\leq 0\\
            (a\cdot b)^2 &\leq ||a||^2||b||^2\\
            |a\cdot b| &\leq ||a||\cdot||b||
        \end{align*}
        \item 
        \begin{align*}
            ||a+b||^2 &= ||a||^2 + 2a\cdot b + ||b||^2\\
                    &\leq ||a||^2 + 2|a\cdot b|| + ||b||^2\\
                    &\leq ||a||^2 + 2||a||\cdot||b|| + ||b||^2\\
                    &= (||a|| + ||b||)^2
        \end{align*}
        \[||a+b||^2 \leq (||a|| + ||b||)^2 \implies ||a+b|| \leq ||a||+||b||\]
        \item \textbf{Exercise.} We want to prove $||a|| = 0 \iff a = 0$ and $||ca|| = |c|\cdot||a||$.
    \end{enumerate}
\end{proof}
%----------------------------Lecture 9 ----------------------------------
\chapter{$\real^d$}
\textbf{Recall:} $||(x_1, \dots, x_d)|| \coloneqq \sqrt{x_1^2 + \cdots + x_d^2}$. This is a norm. i.e. 
$$||a + b|| \leq ||a||_2 + ||b||_2 \ \forall a,b \in \real^d$$ 
$$||ca|| = |c|\cdot ||a||_2, \ c \in \real, \ a \in \real^d$$ 
$$||a||_2 > 0, \ \forall a \in \real^d \setminus \{(0,\dots,0)\}$$
$$||(0,\dots, 0)||_2 = 0$$
Other examples of norms: 
\begin{itemize}
    \item $||(x_1, \dots, x_d)||_1 \coloneqq |x_1| + \cdots + |x_d|$
    \item  $||(x_1, \dots, x_d)||_\infty \coloneqq max\{|x_1|, \dots, |x_d|\}$
\end{itemize}
\textbf{Exercise:} For $a \in \real^d$, $$||a||_\infty \leq ||a||_2 \leq ||a||_1 \leq d||a||_\infty (\leq d||a||_2)$$
\textbf{Interesting Fact:} There are other norms. but they are all equivalent in the sense that if $||\cdot||, ||\cdot||'$ are norms on $\real^d$, then $\exists v, R > 0$ such that
$$r||a|| \leq ||a||' \leq R||a||$$
\section{Convergence}
\begin{definition}
Let $(a_n)_{n=1}^\infty$ be a sequence in $\real^d$ and let $L \in \real^d$, we say $(a_n)_{n=1}^\infty$ \textbf{converges} to $L$, and write $\lim_{n \rightarrow \infty} = L$ or $a_n \rightarrow \infty$, if 
$$\lim_{n \rightarrow \infty} ||a_n - L||_2 = 0$$
\end{definition}
\textit{Note: We could define convergence instead using some other norm, say $||\cdot||_1$.}\\[2ex]
If $||a_n - L||_2 \rightarrow 0$, then $||a_n - L||_1 \leq d||a_n - L||_2 \rightarrow 0$
If $||a_n - L||_1 \rightarrow 0$, then $||a_n - L||_2 \leq d||a_n - L||_1 \rightarrow 0$\\[2ex]
in general, if $||\cdot||$ is any norm, then since $||\cdot||$ and $||\cdot||_2$ are equivalent.
$$||a_n - L||_2 \rightarrow 0 \iff ||a_n - L|| \rightarrow 0$$
\textbf{Example:} Say $a_n = \left(\frac{1}{n}, \frac{1}{n}, \dots, \frac{1}{n}\right)$, then 
$$||a_n - L||_2 = \sqrt{1/n^2, + \cdots + 1/n^2} = \sqrt{\frac{d}{n^2}} = \frac{\sqrt{d}}{n}\rightarrow 0$$
$$\therefore a_n \rightarrow L$$
Given a sequence $(a_n)_{n=1}^\infty$ in $\real^d$, we write $a_1 = (a_1^{(1)},a_1^{(2)}, \dots, a_1^{(d)})$ where $a_1^{(1)},a_1^{(2)}, \dots, a_1^{(d)} \in \real$. Similarily, 
$$a_1 = (a_1^{(1)},a_1^{(2)}, \dots, a_1^{(d)})$$
$$a_2 = (a_2^{(1)},a_2^{(2)}, \dots, a_2^{(d)})$$
$$a_3 = (a_3^{(1)},a_3^{(2)}, \dots, a_3^{(d)})$$
$$\vdots$$
$$L = (L^{(1)},L^{(2)}, \dots, L^{(d)}) \in \real^d$$
We get $d$ sequences in $\real$, and $d$ possible limit points $L^{(1)}, \dots, L^{(d)} \in \real$
\begin{prop}
Given $(a_n)_{n=1}^\infty$ and $L$ as above, $a_n \rightarrow L$ as $d \rightarrow \infty$ $\iff$ $a_n^{(i)} \rightarrow L^{(i)}$ in $\real$ as $n \rightarrow \infty$, for $i = 1, \dots, d$.
\end{prop}
\begin{proof}
$\implies$: Suppose $a_n \rightarrow L$, i.e.
\begin{align*}
    ||a_n - L||_2 \rightarrow 0\\
    \intertext{For $x = (x_1, \dots, x_d) \in \real^d$}
    ||x||^2_2 = x_1^2 + \cdots x_d^2 \geq x_i^2 = |x_i|^2
    \therefore |x_i \leq ||x||_2
\end{align*}
Applying this to (*), we get 
$$0 \leq |a_n^i - L^{(i)}| \leq ||a_n - L ||_2 \rightarrow 0$$
So by the squeeze theorem, 
$$|a_n^{(i)} - L^{(i)}| \rightarrow 0$$\\[2ex]
$\implies$: Suppose $a_n^{(i)} \rightarrow L^{(i)}$ for $i = 1, \dots, d$.
$$||a_n - L||^2_2 = (a_n^{(i)} - L^i)^2 + \cdots + (a_n^d - L^(d)) \rightarrow 0$$
By algebra of limits,
$$\therefore ||a_n - L||_2 \rightarrow 0$$
\end{proof}
\textbf{Example:} $a^n = ((-1)^n, \frac{1}{n}) \in \real^2$. Does $(a_n)$ converge? No, since $(-1)^n$ does not converge.

\begin{definition}
    A sequence $(a_n)_{n=1}^\infty$ in $\real^d$ is \textbf{Cauchy} if  $\forall \epsilon > 0$, $\exists n_0 \in \nat_{\geq 1}$ such that 
    $$||a_n - a_m||_2 < \epsilon \ \forall m,n \geq n_0$$
\end{definition}

\begin{theorem}[Cauchy Convergence Criterion for $\real^d$]
Let $(a_n)_{n=1}^\infty$ be a sequence in $\real^d$. It converges $\iff$ it is Cauchy.
\end{theorem}

\begin{proof}
    $\implies$: Suppose $a_n \rightarrow L \in \real^d$. To show it is Cauchy, let $\epsilon > 0$. $||a_n - L||_2 \rightarrow 0$, so $\exists n_0 \in \nat_{\geq 1}$ such that 
    $$||a_n - L||_2 < \frac{\epsilon}{2} \ \forall n \geq n_0)$$
    Then if $m,n \geq n_0$, 
    \begin{align*}
        ||a_m - a_n||_2 &= ||a_m - L + L - a_n||_2\\
        &\leq ||a_m - L||_2 + ||L - a_n||_2\\
        &< \frac{\epsilon}{2} + \frac{\epsilon}{2} = \epsilon
    \end{align*}
        $\therefore (a_n)_{n=1}^\infty$ is Cauchy.
    $\implies$: Suppose $(a_n)_{n=1}^\infty$ is Cauchy, write 
    $$a_n = (a_n^{(1)}, \dots, a_n^{(d)}$$
    For any $m,n \in \nat_{\geq 1}$,
    $$|a_n^{(i)} - a_m^{(i)} \leq ||a_n - a_m||_2$$
    $\therefore (a_n^{(i)})_{n=1}^\infty$ is Cauchy in $\real$. So by the Cauchy Convergence Criterion, $\exists L^{(i)} \in \real$, such that $a_n^{(i)} \rightarrow L^{(i)}$. By the previous proposition,
    $$a_n \rightarrow (L^{(1)}, \dots, L^{(d)})$$
\end{proof}
\begin{definition}
    $S \subseteq \real^d$ is \textbf{bounded} if $\exists M > 0$ such that 
    $$||x|| \leq M \ \ \forall x \in S$$
\end{definition}
A sequence $(a_n)_{n=1}^\infty$ in $\real^d$ is bound if
$\{a_n: n \in \nat_{\geq 1}\}$ is a bounded set.
\begin{theorem}[Bolzano-Weierstrass for $\real^d$]
    If $(a_n)_{n=1}^\infty$ is a bounded sequence in $\real^d$, then it has a subsequence $(a_{n_k})_{n=1}^\infty$ that converges.
\end{theorem}
\begin{proof}
    Write $a_n = (a_n^{(1)}, \dots, a_n^{(d)})$.\\ 
    We will prove it by indunction on $d$. For $d = 1$, this is the Bolzano-Weierstrass theorem for $\real$ For $d>1$ write
    \[b_n \coloneq (a_n^{(1)}, \ldots, a_n^{(d-1)}) \in \real^{d-1}\]
    By the induction hypothesis, $b_n$ has a subsequence $(b_{n_k})_{k=1}^\infty$ that converges. Let $L \in \real^{d-1}$ be the limit of this subsequence. Then $L \in \real^{d-1}$ is the limit of $b_n$. $(a^{(d)}_{n_k})^\infty_{k=1}$ is a bounded sequence in $\real$, so it has a subsequence $(a^{(d)}_{n_{k_j}})^\infty_{j=1}$ that converges. Let $L^{(d)} \in \real$ be the limit of this subsequence. Then $L = (L^{(1)}, \ldots, L^{(d-1)}, L^{(d)})$ is the limit of $a_n$.
\end{proof}
\chapter{Open and Closed Sets in $\real^d$}
Roughly, an open set is one that we draw with dotted lines. The line represents a "boundary" that is the not in the set. This is not a rigorous definition.
\begin{definition}[Open Ball]
    Let $a\in \real^d$, $r > 0$. The \textbf{open ball} of radius $r$ centered at $a$ is $$B(a;r) \coloneqq \{x \in \real^d: ||x-a||_2 < r\}$$
\end{definition}
\textbf{Relation to Convergence:} If $a_n \rightarrow L$, then this means that $||a_n - L ||_2 < \epsilon$ for all $n$ large. So, $a_n \in B(L;\epsilon)$
\begin{definition}[Open Sets]
    A set $U \subseteq \real^d$ is \textbf{open} if $\forall a \in U$, $\exists r > 0$, such that $B(a; r) \subseteq U$
\end{definition}
\textbf{Idea:} If $a \in U$, then $a$ is not on the boundary but it is truly "inside" the set, so we can fit a ball containing a in the set.
\begin{definition}[Closed Sets]
    A set $k \in \real^d$ is \textbf{closed} if its complement $\real^d \setminus k$ is open. 
\end{definition}
\textbf{Example:} $U \subseteq (0,1)$. Is this open? Yes.
\begin{proof}
    Let $a \in U$. We let $r := \min\{|a-0|, |a-1|\}$ (We do this so that $r$ is at most the distance to the closest bound, i.e. if $a$ is closer to $0$, then the radius $r$ cannot be $|a-1|$)then 
    $$B(a ; r) = (a-r,a+r) \subseteq (0,1) = U$$
\end{proof}
\textbf{Example:} $U \coloneqq [0,1]$. Is this open? No.
\begin{proof}
    Let $a \coloneqq 0 \in U$. The for any $r > 0$, $\exists z \in B(a ; r) = (-r,r)$ s.t $z < 0$, so $z \not\in U$. Therefore $B(a;r) \subseteq U$
\end{proof}
Is $U$ closed? This is the same as asking if $\real \setminus U = (-\infty, 0) \cup (1,\infty)$ is open. 
This is open.
\begin{proof}
    Let $a \in (-\infty, 0) \cup (1,\infty)$.
    \begin{itemize}
        \item \textbf{Case 1:} $a \in (-\infty, 0)$. Set $r := |a|$, so 
        $$B(a;r) = (a-r,a+r) = (2a, 0) \subseteq U$$
        \item \textbf{Case 2:} $a \in (1, \infty)$ similar.
    \end{itemize}
\end{proof}
Therefore $U = [0,1]$ is closed.\\[3ex]
\textbf{Example:} Is $U \coloneqq (0,1]$ open? No, for any $r > 0$
$$B(1;r) \not\subseteq U$$
Therefore, it is not open.\\[2ex]
\begin{center}
    \textit{\textbf{Note:} Sets are not always open or closed. Most sets are neither open nor closed.}
\end{center}
This set $U$ is one such example $U$ is not closed since $\real \setminus U = (-\infty, 0] \cup (1 \infty)$
$0 \in \real \ U$ but $\forall r >0$, $B(0;r) \not\subseteq R \setminus U$\\[3ex]
\textbf{Example:} For any $a \in \real^d$, $r > 0$ $B(a;r)$ is an open set.
\begin{proof}
    Let $x \in B(a;r)$, so $||x-a||_2 < r$. Set 
    $$r_0 \coloneqq r - ||x-a||_2 > 0$$
    \textbf{Claim:} $B(x; r_0) \subseteq B(a;r)$
    To see this, let $y \in B(x;r_0)$ so $|y-x||_2 < r_0$. So,
    \begin{align*}
        ||y-a||_2 &\leq ||y-x||_2 + ||x - a||_2\tag{$\triangle$-inequality} \\
        &< r_0 + ||x - a||_2\\
        &= r
    \end{align*}
\end{proof}
\begin{prop}
    \begin{enumerate}[label=(\roman*)]
        \item $\emptyset$, $\real^d$ are both open in $\real^d$
        \item If $U_1, U_2, \dots, U_n \subseteq \real^d$ are all open, then so is $U_1 \cap U_2 \cap \cdots \cap U_n$.
        \item If $U_a \subseteq \real^d$ is an open set for all $\alpha \in I$, (I is some index set) then 
        $$\bigcup_{a\in I} U_a$$
        is open.
    \end{enumerate}
\end{prop}
\begin{proof}
    \textit{(i),(ii) are exercises}.\\[2ex]
    \textbf{(iii):} Set 
    $$V \coloneqq \bigcup_{\alpha \in I} U_a$$
    Let $a \in V$. This means $\exists \alpha \in I$ such that $a \in U_\alpha$. $U_\alpha$ is open so $\exists r > 0$ s.t $B(a;r) \subseteq U_\alpha$. $U_\alpha \leq \bigcup_{\alpha \in I} U_\alpha = V$ So $B(a;r) \subseteq V$ as required.
\end{proof}
\noindent
\textbf{Example:} For any $n \in \nat_{\geq 1}$.
$$\left(\frac{-1}{n},\frac{1}{n}\right) = B(0;\frac{1}{n})$$ is open in $\real$. The intersection of these open sets is 
$$\bigcap^\infty_{n = 1} \left(\frac{1}{n},\frac{-1}{n}\right) = \{0\} $$ which is not open. This shows that openess is not preserved by infinite intersections. \\[2ex]
\textbf{Example:} Let $$U \coloneqq \{(x,y) \in \real^2 : x > 0 , y >0\}$$
$U$ is open but not closed.\\[1ex]
$$U = V \cap W$$
where 
\begin{align*}
    V \coloneqq \{(x,y) : x > 0\} &&
    W \coloneqq \{(x,y): y >0\}
\end{align*}
To show $V$ is open, let $a = (x,y) \in V$. Set $r \coloneqq x > 0$. Then if $(w,z) \in B(a;r)$. Then 
$$|w-z| \leq ||(w,z)-a||_2 < r = x$$
$$\therefore w > x -x = 0$$
So $(w,z) \in U$. Similarly, $W$ is open. Therefore $U$ is open.\\[2ex]
\textbf{Not Closed:} Exercise.
\begin{prop}
    Let $K \subseteq \real^d$. $K$ is closed $\iff$ for any subsequence $\infseq{a}$ in $K$, If it converges, then
    $$lim_{n \rightarrow \infty} a_n \in K$$
\end{prop}
\begin{proof}
    ($\implies$) Suppose $K$ is closed. Let $\infseq{a}$ be a sequence in $K$ s.t 
    $$L \coloneq lim_{n \rightarrow \infty} a_n$$
    exists. Suppose for a contradction $L \not\in K$. This means $L \in \real^d \setminus K$, which is open. So $\exists r > 0$ such that 
    $$B(L;r) \subseteq \real^d \setminus K$$
    Since $a_n \rightarrow L$, we must have $a_n \in B(L ; a)$ for some $n$ (in fact, for all n sufficiently large. So $a_n \in B(L;r) \subseteq \real^d \setminus K$. Therefore $a_n \not\in K$, which is a contradiction. \\[1ex]
    ($\impliedby$) Suppose $K$ is not closed, and we'll prove $\exists \infseq{a}$ in $K$ such that $a_n \rightarrow L \not\in K$. Since $K$ is not closed, $\real^d \setminus K$ is not open. So $\exists L \in \real^d \setminus K$ such that $\forall r > 0$
    $$B(L;r) \not\subseteq \real^d \setminus K$$
    For each $n \in \nat_{\geq1}$, we can fine $a_n \in B(L; \frac{1}{n})$ such that $a_n \not\in \real^d \setminus K$. So $a_n \in K$. This gives a sequence $\infseq{a}$ in $K$ and 
    $$||a_n - L||_2 < \frac{1}{n} \rightarrow 0$$
    Therefore by the Squeeze Theorem,
    $$||a_n - L||_2 \rightarrow 0 \implies a_n \rightarrow L$$
    $L \in \real^d \setminus K$, so $L \not\in K$.
\end{proof}
\begin{definition}
    Let $A \subseteq \real^d$ and let $a \in \real^d$, a is: 
    \begin{enumerate}[label=(\roman*)]
        \item an \textbf{interior point} if $\exists r > 0$ s.t $B(a;r) \subseteq A$
        \item an \textbf{accumulation point} if $\exists$ a sequence $\infseq{a}$ in $A$ s.t $a_n \rightarrow a$
        \item a \textbf{boundary point} if it is an accumulation point and it is not an interior point.
        $$A^\circ \coloneqq \{All \ interior \ points\}$$
        $$\bar{A} \coloneqq \{All \ accumulation \ points\}$$
        $$\partial A \coloneqq \{All \  boundary 
 \ points\} = \bar{A} \setminus A^\circ$$
    \end{enumerate}
    \begin{center}
        \textit{\textbf{Note:} The set of interior points, accumulation points, and boundary points are referred to as the \textbf{interior} of A, the \textbf{closure} of A, and the \textbf{boundary} of A respectively}
    \end{center}
\end{definition}
\textbf{Example:} $A \coloneqq (0, 1] \cup \{2\}$\\[1ex]
$$A^\circ = (0,1)$$
$$\bar{A} = [0,1] \cup \{2\}$$
$$\partial A = \{0,1,2\}$$
\textbf{Example:} $A \coloneqq \mathbb{Q}$\\[1ex]
Since any open interval contains irrational numbers, we have
$$A^\circ = \empty set$$
Proposition from chapter 2,
$$\bar{A} = \real$$
$$\partial A = \real$$

\chapter{Compactness}
\begin{definition}
    A set $A \subseteq \real^d$ is (sequentially) compact if every sequence $\infseq{a}$ in $A$ has a subsequence $(a_{n_k})^\infty_{k = 1}$ that converges to a point in $A$.
\end{definition}
\textbf{Example 1:} Is [0,1] compact? Yes.
\begin{proof}
    \textbf{Recall:} Bolzano-Weierstrass theorem states bounded sequence has a convergent subsequence. \\[1ex]
    Therefore, every sequence $\infseq{a}$ in $[0,1]$ has a subsequence $\seq{a_n}{k}{\infty}$ that converges. So
    $$0 \leq a_{n_k} \leq 1 \implies 0 \leq \lim_{k \rightarrow \infty} a_{n_k} \leq 1$$
    $$\therefore L \in [0,1]$$
\end{proof}
\textbf{Example 2:} Is (0,1) compact? No.
\begin{proof}
    By counter example, let $a_n \coloneqq \frac{1}{n+1}$, so $a_n \rightarrow 0$. Therefore
    for all subsequences of $a_n$, $a_{n_k} \rightarrow 0$. So there exists no subsequence which converges
    to a point in (0,1).
\end{proof}
\textbf{Example 3:} Is $[0,\infty)$ compact? No.
\begin{proof}
    The Bolzano-Weierstrass theorem does not apply since $[0,\infty)$ is unbounded.
    Set $a_n \coloneqq n$, then $a_n \rightarrow \infty$, so it has no bounded subsequence
    and therefore no convergent subsequences.
\end{proof}
\begin{theorem}[Heine-Borel]
    Let $A \subseteq \real^d$. $A$ is compact $\iff$ A i$s$ closed and not bounded.
\end{theorem}
\begin{proof}
    ($\implies$) Similar to example 1. Assume $A$ is closed and bounded. Let $\infseq{a}$ be a sequence in $A$.
    The sequence is bounded since $A$ is, so by the Bolzano-Weierstrass theorem for $\real^d$, it has a subsequence $\seq{a_n}{k}{\infty}$ that converges to some $L \in \real^d$. $a_{n_k} \in A$ $\forall k$ and $a_{n_k} \rightarrow L$ and $A$ is closed, so by the sequential characterization of closedness, $L \in A$, therefore $A$ is compact.\\[2ex]
    ($\impliedby$) Assume $A$ is compact. To show $A$ is closed, assume for a contradction that $A$ is not closed. Therefore there exists a sequence $\seq{a}{n}{\infty}$ in $A$ such that $a_n \rightarrow L\not\in A$. Then for any subsequence $\seq{a_n}{k}{\infty}$, we have 
    $$a_n \rightarrow L \not\in A$$
    This contradicts that $A$ is compact, therefore $A$ is closed. \\[1ex]
    Tow show $A$ is bounded, assume for a contradiction that $A$ is not bounded. Then $\forall n \in \nat_{\geq1}$, there exists $a_n \in A$ such that $||a_n||_2 \geq n$. This gives a sequence $\seq{a}{n}{\infty}$. Since $A$ is compact, it has a subsequence $\seq{a_n}{k}{\infty}$ that converges. But 
    \[||a_{n_k}||_2 \geq n_k \rightarrow \infty\]
    So $\seq{a_n}{k}{\infty}$ is unbounded, which is a contradiction.
\end{proof}
\begin{prop}
~\newline
    \begin{enumerate}[label=(\roman*)]
        \item If $k_1, \ldots, k_n \subseteq \real^d$ are compact, then $\bigcup_{i=1}^n k_i$ is compact.
        \item If $k_1, \ldots, k_n \subseteq \real^d$ are compact, then $\bigcap_{i=1}^n k_i$ is compact.
    \end{enumerate}
\end{prop}
\begin{proof}
    \textbf{Exercise:}
    \begin{enumerate}[label=(\roman*)]
        \item Assume $A \coloneqq \bigcup_{i=1}^n k_i$. Let $\infseq{a}$ be a sequence in $A$. Then there exists $i \in \{1, \ldots, n\}$ such that $a_n \in k_i$. Since $k_i$ is compact, it has a subsequence $\seq{a_n}{k}{\infty}$ that converges to some $L \in \real^d$. $L \in k_i$ and $k_i \subseteq A$, so $L \in A$. Therefore $A$ is compact.\\[2ex]
        \item Assume $A \coloneqq \bigcap_{i=1}^n k_i$. Let $\infseq{a}$ be a sequence in $A$. Then $a_n \in k_i$ $\forall i \in \{1, \ldots, n\}$. Since $k_i$ is compact, it has a subsequence $\seq{a_n}{k}{\infty}$ that converges to some $L \in \real^d$. $L \in k_i$ $\forall i \in \{1, \ldots, n\}$ and $k_i \subseteq A$, so $L \in A$. Therefore $A$ is compact.
    \end{enumerate}
\end{proof}
\begin{definition}
    $A \subseteq \real^d$ is \textbf{compact} if for any collection
    \[U_\alpha: \alpha \in I\]
    of open sets such that 
    \[A \subseteq \bigcup_{\alpha \in I} U_\alpha\]
    There exists finitely many indeces $\alpha_1, \ldots, \alpha_n$ such that 
    \[A \subseteq \bigcup_{i=1}^n U_{\alpha_i}\]
\end{definition}
\chapter{Limits of a Function of Continous Variables}
A sequence is a function $\nat \rightarrow \real$. Here, we'll consider function 
that are going from $\real \rightarrow \real$ (or $\real^d \rightarrow \real^m$).
\begin{definition}
    Let $X \subseteq \real^d$, $a \in \real^d$ a limit point of $X$.
    $f: X \rightarrow \real^m$, $L \in \real^m$. We say the limit of $f$
    as $X$ approaches $a$ is $L$ if 
    $$\forall \epsilon > 0, \exists \delta > 0 \ s.t \ \forall x \in X$$
    $$x\in B(a;\delta) \wedge x \neq a \implies ||f(x)-L||_2 < \epsilon$$
\end{definition}

The idea is like the definition of convergence of a sequence, except we replace $n \geq n_0$
(which captures "$n$ is sufficiently large") with $x \in B(a;\delta)$, $x \neq a$ (which captures
"$x$ is close to, but not equal to $a$). In other words, the definition says
that if $x$ is close to (but not equal to) $a$
 then $f(x)$ is close to $L$.\\[2ex]

 \textbf{Why "not equal to"?:} Often we consider the limit as $x$ approaches
 $a$ when $f(a)$ is not defined. Other times we compare the limite to $f(a)$.
 So we do not want to use $f(a)$ in the definition of the limit. \\[2ex]

 \textbf{Notation:} We write 
 $$\lim_{x \rightarrow a} f(x) = L$$
 or 
 $$f(x) \rightarrow L \ as \ x \rightarrow d$$
 to mean that the limit of $f$ is $L$ as $x$ approaches $a$.\\[2ex]
 \textbf{Example:} $f: \real \rightarrow \real$. $f(x) \coloneqq 3x - 2$. 
 Let $a \in \real$. Claim
 $$\lim_{x\rightarrow a} f(x) = 3a - 2$$
 \begin{proof}
    Let $\epsilon > 0$. Consider 
    \begin{align*}
        |f(x) - (3a-2)| &= |3x-2 - 3a + 2|\\
        &= 3|x-a|
    \end{align*}
    We want this $< \epsilon$, set $\delta \coloneqq \frac{\epsilon}{3}$. Then if
    $x \in B(a; \delta) = (a - \delta, a + \delta)$ (i.e $|x-a| < \delta$)
    then 
    $$|f(x) - (3a - 2)| = 3|x-a| < 3 \delta = \frac{3\epsilon}{3} = \epsilon$$
 \end{proof}
\textbf{Example:} $g: \real \rightarrow \real$. $g(x) \coloneqq x^2$. Claim: 
$$\lim_{x\rightarrow a} g(x) = a^2$$
\begin{proof}
    Let $\epsilon > 0$ be given.
    \begin{align*}
        |g(x) - a^2| &= |x^2 - a^2| \\
        & = |x - a| |x + a|
    \end{align*}
    What happens if $x$ is close to $a$? Intuitively, $|x + a|$ is close to $|a+a|$
    and $|x-a|$ is small.
    \begin{align*}
        |x+a| &= |x -a + a + a| \leq |x - a| + |a + a|\\
              &< 2|a| + \delta \tag{if $|x-a|<\delta$}\\
              &\leq 2|a| + 1\tag{$\delta \leq 1$}
    \end{align*}
    Then, 
    \begin{align*}
        |x^2 - a^2| &= |x-a||x+a|\leq |x-a|(2|a| + 1)\\
                    &< \delta(2|a| + 1)\tag{if $|x-a|<\delta$}\\
                    &\leq \epsilon\tag{if $\delta \leq \frac{\epsilon}{2|a| + 1}$}
    \end{align*}
    \begin{center}
        \red{\textbf{Important:} Do not define $\delta$ in terms of $x$ or $\delta$!
        We can use $a$ here since $a$ is constant}
    \end{center}
    So we set $\delta \coloneqq \min \{1, \frac{\epsilon}{2|a| + 1}\}$
    Then $\delta \leq \frac{\epsilon}{2|a| + 1}$ and $\delta \leq 1$. So
    if $|x-a| < \delta$. Then from the work above, $|x^2-a^2| < \epsilon$ as required.
\end{proof}

\textbf{Note:} In proofs where we have $\delta - \epsilon$, we often use
$$\delta \coloneqq \min \{\ldots\}$$
In proofs where we have $n_0 - \epsilon$, we often use
$$n_0 \coloneqq \max \{\ldots\}$$
\begin{prop}[Uniqueness of Limits]
    Let $f: X \rightarrow \real^m$ ($X \subseteq \real^d$), $a \in \real^d$
    a limit point of $X$, $L, L' \in \real^m$. If the limit of $f$ as $x \rightarrow a$ is L
    and the limit of $f$ as $x \rightarrow a$ is $L'$, then $L = L'$
\end{prop}
\begin{proof}
    By contradction. Suppose $L \neq L'$. So 
    $$||L - L'||_2 > 0$$
    Set 
    $$\epsilon \coloneqq \frac{||L - L||_2}{2} > 0$$
    Since $f(x) \rightarrow L$ as $x \rightarrow a$, $\exists \delta > 0$
    such that if $x \in X \cap B(a; \delta)\setminus \{a\}$, then
    $$||f(x) - L||_2 < \epsilon$$
    Since $f(x) \rightarrow L'$ as $x \rightarrow a$, $\exists \delta' > 0$ such that
    $$x \in X \cap B(a;\delta')\setminus \{a\} \implies ||f(x) - L'||_2 < \epsilon$$
    Let $\delta_0 \neq \min \{\delta, \delta'\}$. Let 
    $$x \in X \cap B(a;\delta_0) \setminus \{a\}$$
    Then
    $$x \in X \cap B(a; \delta) \setminus \{a\}$$
    So,
    \begin{align*}
        ||f(x) - L||_2 &< \epsilon\\
        ||f(x) - L'||_2 &< \epsilon
    \end{align*}
    So, 
    \begin{align*}
        ||L - L'||_2 \leq ||L - f(x)||_2 + ||f(x) - L'|| &< \epsilon + \epsilon\\
        &= ||L - L'||_2
    \end{align*}
    And thus, a contradction. 
\end{proof}
\begin{prop}[Sequential Characterization of Limits]
    Let $X \subseteq \real^d$, $a \in \real^d$, a limit point of $X$.
    $f: X \rightarrow \real^m$, $L \in \real^m$.\\[1ex]
    $\lim_{x\rightarrow a} f(x) = L \iff$ for every sequence $\infseq{x}$ in $X$
    such that $x_n \rightarrow a$, we have 
    $$\lim_{n \rightarrow \infty} f(x_n) = L$$
    
\end{prop}
\begin{proof}
    ($\implies$) Suppose $\lim_{x\rightarrow a} f(x) = L$. Let $\infseq{x}$ be a sequence in
    $X \setminus \{a\}$ such that $x_n \rightarrow a$. We must show that $f(x_n) \rightarrow L$.\\[1ex]
    Let $\epsilon > 0$ be given. Since $f(x) \rightarrow L$ as $x \rightarrow a$, $\exists \delta$
    such that 
    $$x \in X \cap B(a; \delta) \setminus \{a\} \implies ||f(x) - L||_2 < \epsilon$$
    Since $x_n \rightarrow$, using $\delta$ in place of $\epsilon$, $\exists n_0$ such that
    $\forall n \geq n_0$, $||x_n a||_2 < \delta$. i.e. $x_n \in B(a,\delta)$. Also 
    $x_n \in X \setminus \{a\}$ Therefore, 
    $$||f(x) - L ||_2 < \epsilon$$
    ($\impliedby$) Suppose $\forall$ sequences $\infseq{x}$ ins $X \setminus \{a\}$ converging to $a$, $f(x) \rightarrow L$,
    and for a contradction, suppose $$f(x) \not\rightarrow L$$
    We negate "$f(x) \rightarrow L$" to get that $\exists \epsilon > 0$ such that $\forall \gamma > 0$,
    $\exists x \in X \cap B(a;\delta) \setminus \{a\}$ such that $||f(x) - L||_2 \geq \epsilon$.\\[1ex]
    This gives a sequence $\infseq{x_n}$ in $X \setminus \{a\}$, $||x_n - a||_2 \leq \frac{1}{n}$ $\forall n$,
    so by the squeeze theorem 
    $$||x_n - a||_2 \rightarrow 0$$
    Since $||f(x_n) - L||_2 \geq \epsilon$, $f(x_n) \not\rightarrow L$. This is a contradction.
\end{proof}
\begin{center}
    \textbf{Note:} if $\lim_{n\rightarrow \infty} f(x_n) = L$ for \emph{some} sequence
    $\infseq{x}$ in $X \setminus \{a\}$ convering to a, it \emph{does not} follow
    that $\lim_{x \rightarrow a} f(x) = L$
\end{center}
\textbf{Example:}
$$f(x) \coloneqq\begin{cases}
     0 \text{   if $x = \frac{1}{n}$, $n \in \nat_{\geq 1}$}\\
     1 \text{   otherwise}
\end{cases}$$
$\lim_{x\rightarrow 0}f(x)$ does not exist but $\lim_{n\rightarrow \infty}f(\frac{1}{n}) = 0$
\begin{prop}[Algebra of Limits]
    Let $x \subseteq \real^d$, $a \in \real^d$ a limit point of $X$, $f: X \real^m$
    ,$g: X \rightarrow \real^m$, $L,K \in \real^m$. Suppose $\lim_{x\rightarrow a} f(x) = L$, 
    $lim_{x\rightarrow a} g(x) = K$
    \begin{enumerate}[label=(\roman*)]
        \item $$\lim_{x \rightarrow 0} f(x) + g(x) = L + K$$
        \item $$\lim_{cf(x)} = cL$$
        \item If $m = 1$, $$lim_{x\rightarrow a}f(x)g(x) = LK$$
        \item If $m=1$, $g(x) \neq 0$ $\forall x \in X$, $K \neq 0$. Then
        $$lim_{x\rightarrow a}\frac{f(x)}{g(x)} = \frac{L}{K}$$
    \end{enumerate}
\end{prop}
\begin{proof}
    \begin{enumerate}[label=(\roman*)]
        \item Use Sequential Characterization: Let $\infseq{x}$ be in $X \setminus \{a\}$
        such that $x \rightarrow a$. Then $f(x_n) \rightarrow L$ and $g(x_n) \rightarrow K$
        So by algebra of limits for sequences, 
        $$f(x_n) + g(x_n) = L + K$$
        $$\therefore f(x) + g(x) \rightarrow L + K$$
        \item \textbf{Exercise.}
        \item \textbf{Exercise.}
        \item \textbf{Exercise.}
    \end{enumerate}
\end{proof}
\begin{theorem}[Squeeze Theorem]
    Let $X \subseteq \real^d$, $a \in \real^d$ a limit point of $X$, 
    $f,g,h: X \rightarrow \real$
    $$f(x) \leq g(x) \leq h(x) \ \forall x \in X$$
    and 
    $$\lim_{x\rightarrow a} f(x) =  \lim_{x \rightarrow a} h(x) = L$$
    Then
    $$\lim_{x\rightarrow a} g(x) = L$$
\end{theorem}
\begin{proof}
    \textbf{Exercise}
\end{proof}
If $f: X \rightarrow \real_m$, We can define functions 
$$f_1, \ldots, f_m: X \rightarrow \real$$
by 
$$(f_1(x), \ldots, f_m(x)) = f(x)$$
$f_1, \ldots, f_m$ are callF    ed the \emph{component functions} of $f$.
\begin{prop}
    Let $X \in \real^d$, $a \in \real^d$ a limit point of $X$, $f: X \rightarrow \real^m$,
    $f_1,\ldots, f_m$ its component functions. $L = (L_1, \ldots, L_m) \in \real^m$. Then
    $$\lim_{x\rightarrow a} f(x) = L \iff \lim_{x\rightarrow a} f_i(x) = L_i \ \forall 1 \leq i \leq m$$
\end{prop}
\begin{proof}
    \textbf{Exercise.}
\end{proof}
\begin{definition}
    Let $X \subseteq \real$, $a \in \real$, $f: X \rightarrow \real^d$.
    \begin{itemize}
        \item If $a$ is a limit point of $X \cap (a, \infty)$ then we write
            $\lim_{x\rightarrow a^+} f(x) = L$ to mean that 
            $$\lim_{x \rightarrow a} g(x) = L$$
            where 
            $$g = f{\big|}_{X \cap (a, \infty)}$$
        \item If $a$ is a limit point of $X \cap (-\infty,a)$ then we write
        $\lim_{x\rightarrow a^+} f(x) = L$ to mean that 
        $$\lim_{x \rightarrow a} g(x) = L$$
        where 
        $$g = f \mid_{X \cap (-\infty, a)}$$
    \end{itemize}
\end{definition}
\textbf{Example:}
$$f(x)\coloneqq \begin{cases}
    -1, x < 0\\
    0, x = 0\\
    1, x > 0
\end{cases}$$
$$\lim_{x \rightarrow 0^+} f(x) = 1 \neq \lim_{x \rightarrow 0^-} f(x) = -1$$
\end{document}