\documentclass[openany]{report}
\usepackage[utf8]{inputenc}

\usepackage{stylesheets}
\usepackage{lecture_notes_styles}
\title{MAT 2125 Lecture Notes}
\author{Last Updated:}

\begin{document}

\maketitle

\tableofcontents


\chapter*{Important Proofs for Midterm}
\label{chapter:midterm}
\addcontentsline{toc}{chapter}{\hyperref[chapter:midterm]{Important Proofs for Midterm}}


\begin{manualtheorem}{1.3.13}[The Archimedean Property]
    The set $\mathbb{N}_{\geq1}$ is not bounded above.
\end{manualtheorem}
    
    \begin{proof}
    Suppose for a contradiction that $\mathbb{N}$ was bounded above. Then by completeness, $a = \sup \mathbb{N}$ exists. Since $a$ is a least upper bound, $a - 1$ is not an upper bound, so there exists $m \in \mathbb{N}$ such that 
    \[m > a - 1\]
    Then since $m \in \mathbb{N}$, we have $m + 1 \in \mathbb{N}$, so 
    \[m + 1 > a\]
    But $a$ is an upper bound, thus a contradiction. 
    \end{proof}
    \begin{manualprop}{2.2.4}[Uniqueness of Limits]
    Let $(a_n)_{n=1}^\infty$ be a sequence and let $L_1, L_2 \in \mathbb{R}$. If 
    \[\lim_{n\rightarrow\infty} a_n = L_1 \text{ and } \lim_{n\rightarrow\infty} a_n = L_2\]
    then 
    \[L_1 = L_2\]
    \end{manualprop}
    \begin{proof}
        Suppose for a contradiction $L_1 \neq L_2$. We can assume without loss of generality that $L_1 < L_2$. Define 
        \[\epsilon = \frac{L_2 - L_1}{2}\]
        Since $\lim\limits_{n\rightarrow\infty} a_n = L$, there exists $n_0$ such that $\forall n \geq n_0$
        \[L_1 - \epsilon < a_n < L_1 + \epsilon\]
        Using the second inequality and the definition of $\epsilon$, we get
        \[a_n < L_1 + \epsilon = L_1 + \frac{L_2 - L_1}{2} = L_1 + \frac{L_2}{2} - \frac{L_1}{2} = \frac{L_2 + L_1}{2}\]
        Likewise, since $\lim\limits_{n\rightarrow\infty} a_n = L_2$, there exists $m_0$ such that for all $n \geq m_0$, 
        \[L_2 - \epsilon < a_n < L_2 + \epsilon\]
        Then from the first inequality, we get
        \[a_n > L_2 - \epsilon = L_2 - \frac{L_2 - L_1}{2} = \frac{L_2 + L_1}{2}\]
        So, we get that for all $n \geq \max\{n_0, m_0\}$, 
        \[a_n > \frac{L_2 + L_1}{2} > a_n\]
        Thus, a contradiction.
    \end{proof}
    \begin{manualprop}{2.2.8}
        Let $(a_n)_{n=1}^\infty$ be a sequence which converges to some number $L \in \mathbb{R}$. Then $(a_n)_{n=1}^\infty$ is bounded.
    \end{manualprop}
    \begin{proof}
        Since $\lim\limits_{n\rightarrow\infty} a_n = L$, set $\epsilon \coloneqq 1$, there exists $n_0$ such that for all $n \geq n_0$
        \[|a_n - L| < 1\]
        So we have that $\forall n \geq n_0$
        \[L - 1 < a_n < L + 1\]
        Now set
        \[M \coloneqq \max\{a_1, a_2, \ldots, a_{n_0-1}, L+1\}\]
        If $n < n_0$, then it is amongst the set $\{a_1, \ldots, a_{n_0-1}\}$, so $M$ will be the max of this set. Therefore, $\forall n < n_0$, $a_n \leq M$. Then for $n \geq n_0$, by the definition of the limit we know that $a_n < L + 1$, so we get that $a_n < L + 1 \leq M$. Therefore, for all values of $n$,the set $\{a_n: n \in \mathbb{N}\}$ is bounded above.\\[1ex]
        Similarly for the lower bound, take
        \[M \coloneqq \min\{a_1, a_2, \ldots, a_{n_0 -1}, L - 1\}\]
        If $n < n_0$, then it is in the set $\{a_1, a_2, \ldots, a_{n_0-1}\}$
        $M'$ is at most the minimum of this set, so $\forall n < n_0$, $a_n \geq M'$. If $n \geq n_0$, by the definition of the limit we know that for all $n \geq n_0$, $a_n > L - 1$. So $M'$ is at most $L - 1$. Therefore $\forall n \geq n_0$, $a_n > L - 1 \geq M'$. Therefore, the set is bounded below and above, so it is bounded.
    \end{proof}
    \begin{manualprop}{2.3.3}
        Let $(a_n)_{n=1}^\infty$ and $(b_n)_{n=1}^\infty$ be converging sequences, if
        \[a_n \leq b_n\]
        for all $n$, then
        \[\lim_{n\rightarrow \infty} a_n \leq \lim_{n\rightarrow\infty} b_n   \]
    \end{manualprop}
    \begin{proof}
        Suppose that $(a_n)_{n=1}^\infty$ and $(b_n)_{n=1}^\infty$ are convergent sequences with $a_n < b_n$ for all $n$. Then by the definition of convergence, we have that $\forall \epsilon > 0$, $\exists n_0$ such that $\forall n \geq n_0$
        \[|a_n - L_a| < \epsilon\]
        Similarly for $b_n$, we have that $\exists m_0$ such that $\forall \epsilon > 0$, 
        \[|b_n - L_b| < \epsilon\]
        Now suppose for a contradiction that $L_a > L_b$, then set $\epsilon \coloneqq \frac{L_a - L_b}{2}$. So we have 
        \[L_a-\epsilon < a_n < \epsilon + L_a\]
        So,
        \[a_n > L_a - \epsilon = L_a - \frac{L_a - L_b}{2} = \frac{L_a + L_b}{2}\]
        Similarly for $b_n$, we have 
        \[L_b - \epsilon < b_n < L_b + \epsilon\]
        \[b_n < L_b + \epsilon = \frac{L_a + L_b}{2}\]
        So we have $b_n < \frac{L_b+L_a}{2} < a_n$, but $a_n < b_n$. Thus, a contradiction.
    \end{proof}
    \begin{manualtheorem}{2.3.5}[Squeeze Theorem]
    Let $(a_n)_{n=1}^\infty$, $(b_n)_{n=1}^\infty$, $(c_n)_{n=1}^\infty$ be sequences such that 
    \begin{enumerate}[label=(\roman*)]
        \item $(a_n)_{n=1}^\infty$ and $(c_n)_{n=1}^\infty$ converge to the same number $L$, and 
        \item $a_n \leq b_n \leq c_n$ for all n
        Then $(b_n)_{n=1}^\infty$ also converges to L.
    \end{enumerate}
    \end{manualtheorem}
    \begin{proof}
        Let $\epsilon > 0$ be given. Suppose $a_n \leq b_n \leq c_n$ $(a_n)_{n=1}^\infty$ and $(b_n)_{n=1}^\infty$ converge to $L$, so $\exists n_a, n_c \in \mathbb{N}$ such that for all $n \geq n_a$
        \[L - \epsilon < a_n < L + \epsilon\]
        and 
        \[L - \epsilon < c_n < L + \epsilon\]
        So 
        \[L - \epsilon < a_n \leq b_n \leq c_n < L + \epsilon\]
        Therefore,
        \[L - \epsilon < b_n < L + \epsilon\]
        By the definition of convergence, $(b_n)_{n=1}^\infty$ converges to $L$.
    \end{proof}
    \begin{manualtheorem}{2.6.1}[Cauchy Convergence Criterion]
    Let $(a_n)_{n=1}^\infty$ be a sequence of real numbers. Then it converges if and only if it is Cauchy.
    \end{manualtheorem}
    \begin{proof}
        ($\implies$) Assume that $(a_n)_{n=1}^\infty$ converges, then there exists $n_0$ such that for all $\epsilon > 0$,  $\forall n \geq n_0$
        \[|a_n - L| < \epsilon\]
        Now take $\frac{\epsilon}{2}$ in place of $\epsilon$ since $\epsilon$ is arbitrary, we have
        \[|a_n - L| < \frac{\epsilon}{2}\]
        Then, for $m,n \geq n_0$, we have 
        \[a_m - a_n| = |a_m - L + L - a_n| \leq |a_m - L| + |L - a_n| = |a_m - L| + |a_n - L|\]
        Since $m, n \geq n_0$, by the definition of convergence we have
        \[|a_m - L| + |a_n - L| < \frac{\epsilon}{2} + \frac{\epsilon}{2} = \epsilon\]
        Therefore,
        \[|a_m - a_n| < \epsilon\]
        as required.
        
    \end{proof}


    \begin{manualprop}{2.7.3}
        For any sequence $\infseq{a}$,
        \[\liminf_{n\rightarrow\infty} a_n \leq \limsup_{n\rightarrow \infty} a_n\]
    \end{manualprop}
    \begin{proof}
        If the sequence isn't bounded, then either $\limsup\limits_{n\rightarrow\infty}a_n = \infty$ or $\liminf\limits_{n\rightarrow\infty} a_n = -\infty$, in either case the result is trivial. So assume that the sequence is bounded. Consider the sets used to define $\limsup$ and $\liminf$
        \[S \coloneqq \{\beta : \real : \exists n_0 \text{ such that } a_n \leq \beta \ \  \forall n \geq n_0\}\]
        \[T \coloneqq \{\alpha : \real : \exists m_0 \text{ such that } a_n \geq \alpha \ \ \forall n \geq m_0\}\]
        So we have $\alpha \in T$ and $\beta \in S$, then for all $n \geq \max\{n_0, m_0\}$, we have
        \[\alpha \leq a_n \leq \beta\]
        Thus, we have shown that for every $\alpha \in T$, and every $\beta \in S$, we have $\alpha \leq \beta$. From the definition of $\limsup$ and $\liminf$, we get that for any eventual lower bound $\alpha \in T$, it is a lower bound for the set of upper bounds $S$, so
        \[\alpha \leq \inf T = \limsup_{n\rightarrow\infty} a_n \]
        So then $\limsup\limits_{n\rightarrow\infty}a_n$ is an upper bound for the set of lower bounds $T$, so
        \[\limsup_{n\rightarrow\infty} a_n \geq \sup T = \liminf_{n\rightarrow\infty} a_n\]
        Therefore, 
        \[\liminf_{n\rightarrow\infty} a_n \leq \limsup_{n\rightarrow\infty} a_n\]
        as required.


    \end{proof}


    \begin{manualprop}{}
        The harmonic series 
        \[\sum_{n=1}^\infty \frac{1}{n}\]
        diverges. 
    \end{manualprop}
\begin{proof}
    Consider the partial sum of the series 
    \[S_2 = 1 + \frac{1}{2} = \frac{3}{2}\]
    Now consider the partial sums which correspond to powers of 2, $S_{2^N}$ for $N \in \nat$. So we have the sums $S_2, S_4, S_8,\dots$. Now consider the sequence of partial sums
    \[S_4 = 1 + \frac{1}{2} + \left(\frac{1}{3} + \frac{1}{4}\right)\]
    $\frac{1}{3} > \frac{1}{4}$, so we have that 
    \[S_4 > 1 + \frac{1}{2} + \left(\frac{1}{4} + \frac{1}{4}\right) = 1 + \frac{2}{2}\]
    Continuing similarily,
    \begin{align*}
        S_8 = S_{2^3} &= 1 + \frac{1}{2} + \left(\frac{1}{3} + \frac{1}{4}\right) + \left(\frac{1}{5} + \frac{1}{6} + \frac{1}{7} + \frac{1}{8}\right) > 1 + \frac{1}{2} + \frac{1}{2} + \frac{4}{8} = 1 + \frac{3}{2}\\
        &\vdots\\
        S_{2^N} &> 1 + \frac{N}{2}
    \end{align*}
    So, we have
    \[\lim_{N\rightarrow \infty} \left(1 + \frac{N}{2}\right) = \infty\]
    But, $S_{2^N} > 1 + \frac{N}{2}$ for all $N \in \nat$, so we have that the partial sums diverge. Therefore, the series diverges.
\end{proof}


\begin{manualprop}{3.1.7}[Divergence Test]
    Let $\infseq{a}$ bea sequence of real numbers. If the series
    \[\infser{a}\]
    converges, then
    \[\lim_{n\rightarrow\infty} a_n = 0\]
\end{manualprop}
\begin{proof}
    Suppose $\infser{a}$ converges to $L$. Set $L \coloneqq \infser{a}$. Consider the partial sums
    \[S_N = \sum_{n=1}^N a_n\]
    so $\lim\limits_{n\rightarrow\infty}S_N = L$. We also have that $\lim\limits_{n\rightarrow\infty}S_{N-1} = L$, since 
    \[\lim_{N\rightarrow\infty} S_{N-1} = \lim_{N\rightarrow\infty} \sum_{n=1}^{N-1} a_n = \sum_{n=1}^{\infty-1} a_n = \sum_{n=1}^{\infty} a_n = L\]
    Then, we have that 
    \[S_N - S_{N-1} = \sum_{n=1}^N a_n - \sum_{n=1}^{N-1} a_n = a_N\]
    So, 
    \[\lim_{N\rightarrow \infty} S_N - S_{N-1} = L - L = 0\]
    \[\lim_{N\rightarrow \infty} S_N - \lim_{N\rightarrow\infty} S_{N-1} = \lim_{N\rightarrow \infty} \sum_{n=1}^N a_n - \lim_{N\rightarrow \infty} \sum_{n=1}^{N-1} a_N = \lim_{N\rightarrow \infty} a_n = 0\]
\end{proof}


\begin{manualprop}{3.2.1}[Boundedness Test]
    Let $\infseq{a}$ be a sequence of real numbers. Suppose that
    \begin{enumerate}[label=(\roman*)]
        \item $a_n \geq 0$ for all $n$, and
        \item There is a bound $M \in \real$ on the partial sums, so that
        \[\sum_{n=1}^N a_n \leq M\]
        for all $N \in N_{\geq 1}$.
    \end{enumerate}
    Then $\infser{a}$ converges.
\end{manualprop}
\begin{proof}
    Since $a_n \geq 0$, the partial sums $(S_N)_{N=1}^\infty$ satisfy
    \[S_N \leq S_{N+1} \text{ for all $N$.}\]
    In other words, $(S_N)^\infty_{N=1}$ is an increasing sequence. The second condition ensures that the sequence is bounded above. Therefore, by the Monotone Convergence Criterion, it converges. Therefore, $\sum\limits_{n=1}^\infty a_n$ converges.
\end{proof}

\begin{manualprop}{3.2.2}[Comparison Test]
    Let $\infseq{a}$ and $\infseq{b}$ be sequences of real numbers such that
    \[0 \leq a_n \leq b_n \text{ for all n}\]
    Then,
    \begin{enumerate}[label=(\roman*)]
        \item if $\sum\limits_{n=1}^\infty b_n$ converges, then so does $\sum\limits_{n=1}^\infty a_n$
        \item i1f $\sum\limits_{n=1}^\infty a_n$ diverges, then so does $\sum\limits_{n=1}^\infty a_n$
    \end{enumerate}
\end{manualprop}
\begin{proof}
    Since the sequence $\sum\limits_{n=1}^\infty$ converges, take $M \coloneqq \sum\limits_{n=1}^\infty$. Then, we have the sequence of partial sums 
    \[\left(\sum_{n=1}^\infty b_n\right)_{n=1}^\infty\]
    is increasing and converges to $M$, so $M$ is the supremum of this sequence, therefore
    \[\sum_{N=1}^N b_n \leq M \]
    for all $M$. Therefore
    \[\sum_{N=1}^N a_n \leq \sum_{N=1}^N b_n \leq M\]
    Therefore, by the Boundedness test, $\sum\limits_{n=1}^\infty a_n$ converges. $(ii)$ is the contrapositive of $(i)$ so it follows that it holds. 
\end{proof}

\begin{manualprop}{3.2.3}[Absolute Convergence Test]
    Let $\infseq{a}$ be a sequence of real numbers. If the series 
    \[\infser{|a}|\]
    converges, then so does 
    \[\infser{a}\]
\end{manualprop}
\begin{proof}
    Assume $\infser{|a}|$ converges, Write 
    \[(a_n)_+ = \max\{a_n, 0\}\]
    \[(a_n)_- = \max\{-a_n, 0\}\]
    So $(a_n)_+$ is all the positive terms from $a_n$ and $(a_n)_-$ is all the negative terms from $a_n$, but we are negating them so that they are positive, so we have 
    \[a_n = (a_n)_+ - (a_n)_-\]
    Then, we have that 
    \[0 \leq (a_n)_+ \leq |a_n|\] 
    So, by the Comparison Test, we have that $|a_n|$ converges so $\sum\limits_{n=1}^\infty (a_n)_+$ converges. Similarily, 
    \[0 \leq (a_n)_- \leq |a_n|\]
    Therefore by the Comparison Test, $\sum\limits_{n=1}^\infty (a_n)_-$ converges. So by linearity,
    \[\sum_{n=1}\infty a_n = \sum_{n=1}\infty (a_n)_+ - \sum_{n=1}\infty (a_n)_- \]
    converges.
\end{proof}

\begin{manualprop}{4.2.3}
    Let $\infseq{a}$ be a sequence in $\real^d$, with 
    \[a_n = (a_n^{(1)}, \ldots, a_n^{(d)}) \text { for each } n \in \nat\]
    and let $L = (L_1, \ldots, L_d) \in \real^d$. Then
    \[\lim_{n\rightarrow\infty} a_n = L\]
    if and only if, for each $i = 1, \ldots, d$,
    \[\lim_{n\rightarrow\infty} a_n^{(i)} = L_i\]
\end{manualprop}
\begin{proof}
    ($\implies$) Assume that $\lim\limits_{n\rightarrow\infty} a_n = L$. Then, for each $i = 1, \ldots, d$, we have that $|x_i|^2 \leq \sum_{i=1}^{d} x_i^2 = ||x||_2^2$, therefore 
    \[|x_i| \leq ||x||_2\]
    Using this fact, we then have each component of $||a_n - L||_2$ is less than or equal to it. So 
    \[|a_n^{(i)} - L_i| \leq ||a_n - L||_2\]
    \[-||a_n - L||_2 \leq a_n^{(i)} - L_i \leq ||a_n-L||_2\]
    Since $\lim\limits_{n\rightarrow \infty}a_n = L$, we have 
    $\lim\limits_{n\rightarrow\infty} a_n - L = 0$. By the Squeeze theorem, it follows that 
    \[\lim_{n \rightarrow \infty} a_n^{(i)} - L_i = 0 \implies \lim_{n\rightarrow \infty} a_n^{(i)} = L\]
    ($\impliedby$) Suppose for each $i = 1, \ldots, d$, we have 
    \[\lim_{n\rightarrow\infty} a_n^{(i)} = L_i\]
    Then, from the definition of $||\cdot||_2$, we have 
    \[||a_n -L||_2^2 = (a_n^{(1)} - L_1)^2 + \cdots + (a_n^{(d)} - L_d)^2\]
    Now taking limits of both sides
    \[\lim_{n\rightarrow\infty} ||a_n - L||_2^2 = \lim_{n\rightarrow\infty} (a_n^{(1)} - L_1)^2 + \cdots + \lim_{n\rightarrow\infty} (a_n^{(d)} - L_d)^2\]
    Now we'll prove exercise 2.2.5 which states that if $\infseq{a}$ is a sequence of non-negative real number converging to $L \geq 0$, then $\lim\limits_{n \rightarrow \infty} \sqrt{a_n}$ converges to $\sqrt{L}$. To prove this we will consider two cases where $L = 0$, and $L > 0$. 
    \begin{itemize}
        \item \textbf{Case 1, L = 0}: Suppose $\infseq{a} \rightarrow 0$, then from the definition of convergence we have that $\forall \epsilon > 0$, $\exists n_0$ such that $\forall n \geq n_0$, 
        \[|a_n - 0| < \epsilon\]
        Since $\epsilon$ is abritrary, we'll replace $\epsilon$ with $\epsilon^2$, so 
        \[|a_n - 0| < \epsilon^2\]
        Then we get 
        \[|a_n - 0| = |a_n| < \epsilon^2 \implies \sqrt{|a_n|} < \epsilon\]
        Therefore, $\sqrt{a_n} \rightarrow 0$ by the definition of convergence.
        \item \textbf{Case 2, $L > 0$}: Suppose $\infseq{a} \rightarrow L > 0$. Let $\epsilon >0$ be given, then there exists $n_0$ such that for all $n \geq n_0$,
        \[|a_n - L| < \epsilon\]
        We much such that $|\sqrt{a_n} - \sqrt{L}| < \epsilon$
        \[|\sqrt{a_n} - \sqrt{L}| \cdot \frac{\sqrt{a_n} + \sqrt{L}}{\sqrt{a_n} + \sqrt{L}} = \frac{|(\sqrt{a_n} - \sqrt{L})(\sqrt{a_n} + \sqrt{L})|}{\sqrt{a_n} + \sqrt{L}}\]
        Since $\sqrt{a_n} + \sqrt{L}$ is positive because $a_n, L \geq 0$, then $\sqrt{a_n} + \sqrt{L} = |\sqrt{a_n} + \sqrt{L}|$, then using the fact that $|a|\cdot|b| = |a\cdot b|$, we get 
        \[\frac{|a_n - \sqrt{L}\sqrt{a_n} + \sqrt{L}\sqrt{a_n} + L|}{\sqrt{a_n} + \sqrt{L}} = \frac{|a_n - L|}{\sqrt{a_n} + \sqrt{L}} \leq \frac{|a_n - L|}{\sqrt{L}}\]
        Now if we replace $\epsilon$ with $\epsilon \over \sqrt{L}$, we get 
        \[|\sqrt{a_n} - \sqrt{L}| < \frac{|a_n - L|}{\sqrt{L}} < \frac{\epsilon}{\sqrt{L}} \implies |\sqrt{a_n} - \sqrt{L}| < \epsilon\]
        Therefore, $\lim\limits_{n\rightarrow\infty}\sqrt{a_n} = \sqrt{L}$
    \end{itemize}
    Now going back to the original proof, 
    \[\lim_{n\rightarrow \infty} ||a_n - L||_2^2 = 0\]
    So from exercise 2.2.5 we have
    \[\lim_{n\rightarrow \infty} \sqrt{||a_n - L||^2_2} = \sqrt{0}\]
    Therefore,
    \[\lim_{n \rightarrow \infty} ||a_n - L||_2 = 0\]
    as required.
\end{proof}

\begin{manualtheorem}{4.2.2}[Cauchy Covergence $\real^d$]
    Let $\infseq{a}$ be a sequence $\real^d$. Then it converges if it converges if and only if it is cauchy.
\end{manualtheorem}
\begin{proof}
    Suppose $\infseq{a}$ is a sequence in $\real^d$ that converges to $L \in \real^d$. Let $\epsilon > 0$ be given, then there exists $n_0$ such that $\forall m,n \geq n_0$,
    \[||a_n - L||_2 < \epsilon\]
    \[||a_m - L||_2 < \epsilon\]
    Since $\epsilon$ arbitrary we can replace $\epsilon$ with $\frac{\epsilon}{2}$, so 
    \[||a_n-L||_2 = \frac{\epsilon}{2} \text{ and } ||a_m - L||_2 = \frac{\epsilon}{2}\]
    So,
    \[||a_m - a_n||_2 = ||a_m - L + L - a_n||_2 \leq ||a_m - L||_2 + ||L - a_n||_2\]
    \[= ||a_m - L||_2  + ||a_n - L||_2 < \frac{\epsilon}{2} +  \frac{\epsilon}{2} = \epsilon\]
\end{proof}

\begin{manualprop}{4.3.4}
    Given $a \in \real^d$, and $r > 0$, the open ball $B(a,r)$ is an open set. 
\end{manualprop}
\begin{center}
    \textbf{Note: This is example 4.3.4 from the professors notes.}
\end{center}
\begin{proof}
    Recall the definition of an open set is that for any $x$ in the set, we can define an open ball (or epsilon neighborhood) around $x$ such that the ball is contained in the set. So we want an open ball $B(x;\epsilon)$ such that $B(x;\epsilon) \subseteq B(a;r)$. To see this, let $x \in B(a;r)$, so that $||x-a||_2 < r$. Define 
    \[\epsilon \coloneqq r - ||a-x||_2 > 0\] 
    Now take some element $y \in B(x;\epsilon)$, then we want to show that tis element is contained in $B(a;r)$. So, $y \in B(x;\epsilon)$, so that $||y-x||_2 < \epsilon$. Then,
    \begin{align*}
        ||y-a||_2 = ||y - x + x -a||_2 &\leq ||y - x||_2 + ||x -a||_2\\
        &< \epsilon + ||a-x||_2 = r
    \end{align*}
    Therefore, 
    \[||y-a||_2 < r\]
    So $y \in B(a;r)$ as required, so $B(a;r)$ is an open set.
\end{proof}


\begin{manualprop}{4.3.5}
    \begin{enumerate}[label=(\roman*)]
        \item The sets $\emptyset$, $\real^d$ are open
        \item For any finite collection of open sets, $U_1, \ldots, U_m \subseteq \real^d$, their intersection is
        \[U_1 \cap \cdots \cap U_m\]
        is open
        \item For any arbitrary collection of open sets $\{U_\alpha: \alpha \in I\}$, their union,
        \[\bigcup_{\alpha\in I} U_\alpha\]
        is open.
    \end{enumerate}
\end{manualprop}
\begin{proof}
    \begin{enumerate}[label=(\roman*)]
        \item $(i)$ and $(ii)$ are Exercise 4.3.1
        \item Will add them later!
        \item Set
        \[U \coloneqq \bigcup_{\alpha\in I} U_\alpha\]
        Since $U_\alpha$ is open, there is some $\epsilon > 0$ such that 
        \[B(x;\epsilon) \subseteq U_\alpha\]
        Then since $U$ is the union of all the $U_\alpha$, we have that $U_\alpha \subseteq U$ so it follows that 
        \[B(x;\epsilon) \subseteq U\]
        as required. 
    \end{enumerate}
\end{proof}

\begin{manualtheorem}{4.4.5}[Heine-Borel Theorem]
    Let K be a subset of $\real^d$. Then K is compact if and only if K is closed and bounded.
\end{manualtheorem}
\begin{proof}
   ($\implies$) Suppose that $K$. To see that $K$ is closed, suppose for a contradiction that it is not closed. By proposition 4.3.9, $F$ is closed if and only if for every sequence $\infseq{a}$ in $F$, if $\infseq{a}$ converges then 
   \[\lim_{n\rightarrow \infty} a_n \in F\]
   So, if $K$ is not closed, then it follows that there exists some sequence $\infseq{a}$ such that 
   \[\lim_{n\rightarrow\infty} a_n \not\in K\]
   Then by proposition 2.5.4, if a sequence $\infseq{a}$ converges, then all subsequences of the sequence converge to the same point, and hence no subsequence converges to a point in $K$. This contradicts the fact that $K$ is compact. Similarly for the boundess of $K$, suppose for a contradiction that $K$ was not bounded. Then for any $n \in \nat$, there exists $a_n \in K$ such that $||a_n||_2 \geq n$. So the sequence $\infseq{a}$ is unbounded, as well as all subsequences $(a_{n_k})^\infty_{k=1}$. Therefore, no subsequence converge since they are all unbounded. This contradicts the fact that $K$ is compact.\\[2ex]
   ($\impliedby$) Assume $K$ is closed and bounded. Since $K$ is bounded, any sequence $\infseq{a}$ in $K$ is bounded. Then by the Bolzano-Weierstrass theorem, $\infseq{a}$ is bounded so there exists a convergent subsequence $(a_{n_k})_{k=1}^\infty$ that converges to some $L \in \real^d$. Since $K$ is closed, from proposition 4.3.9 we have that 
   \[\lim_{k\rightarrow\infty} a_{n_k} \in K\]
   Therefore every sequence $\infseq{a}$ has a subsequence $(a_{n_k})_{k=1}^\infty$ that converges to some $L \in K$, so $K$ is compact.
\end{proof}
    
\chapter*{Important Proofs for Final}
\label{chapter:final}
\addcontentsline{toc}{chapter}{\hyperref[chapter:final]{Important Proofs for Final}}


\begin{manualtheorem}{1.3.13}[The Archimedean Property]
    The set $\mathbb{N}_{\geq1}$ is not bounded above.
\end{manualtheorem}
    
    \begin{proof}
    Suppose for a contradiction that $\mathbb{N}$ was bounded above. Then by completeness, $a = \sup \mathbb{N}$ exists. Since $a$ is a least upper bound, $a - 1$ is not an upper bound, so there exists $m \in \mathbb{N}$ such that 
    \[m > a - 1\]
    Then since $m \in \mathbb{N}$, we have $m + 1 \in \mathbb{N}$, so 
    \[m + 1 > a\]
    But $a$ is an upper bound, thus a contradiction. 
    \end{proof}


    \begin{manualprop}{2.2.4}[Uniqueness of Limits]
        Let $(a_n)_{n=1}^\infty$ be a sequence and let $L_1, L_2 \in \mathbb{R}$. If 
        \[\lim_{n\rightarrow\infty} a_n = L_1 \text{ and } \lim_{n\rightarrow\infty} a_n = L_2\]
        then 
        \[L_1 = L_2\]
    \end{manualprop}
    \begin{proof}
            Suppose for a contradiction $L_1 \neq L_2$. We can assume without loss of generality that $L_1 < L_2$. Define 
            \[\epsilon = \frac{L_2 - L_1}{2}\]
            Since $\lim\limits_{n\rightarrow\infty} a_n = L$, there exists $n_0$ such that $\forall n \geq n_0$
            \[L_1 - \epsilon < a_n < L_1 + \epsilon\]
            Using the second inequality and the definition of $\epsilon$, we get
            \[a_n < L_1 + \epsilon = L_1 + \frac{L_2 - L_1}{2} = L_1 + \frac{L_2}{2} - \frac{L_1}{2} = \frac{L_2 + L_1}{2}\]
            Likewise, since $\lim\limits_{n\rightarrow\infty} a_n = L_2$, there exists $m_0$ such that for all $n \geq m_0$, 
            \[L_2 - \epsilon < a_n < L_2 + \epsilon\]
            Then from the first inequality, we get
            \[a_n > L_2 - \epsilon = L_2 - \frac{L_2 - L_1}{2} = \frac{L_2 + L_1}{2}\]
            So, we get that for all $n \geq \max\{n_0, m_0\}$, 
            \[a_n > \frac{L_2 + L_1}{2} > a_n\]
            Thus, a contradiction.
    \end{proof}


        \begin{manualprop}{2.2.8}
            Let $(a_n)_{n=1}^\infty$ be a sequence which converges to some number $L \in \mathbb{R}$. Then $(a_n)_{n=1}^\infty$ is bounded.
        \end{manualprop}
        \begin{proof}
            Since $\lim\limits_{n\rightarrow\infty} a_n = L$, set $\epsilon \coloneqq 1$, there exists $n_0$ such that for all $n \geq n_0$
            \[|a_n - L| < 1\]
            So we have that $\forall n \geq n_0$
            \[L - 1 < a_n < L + 1\]
            Now set
            \[M \coloneqq \max\{a_1, a_2, \ldots, a_{n_0-1}, L+1\}\]
            If $n < n_0$, then it is amongst the set $\{a_1, \ldots, a_{n_0-1}\}$, so $M$ will be the max of this set. Therefore, $\forall n < n_0$, $a_n \leq M$. Then for $n \geq n_0$, by the definition of the limit we know that $a_n < L + 1$, so we get that $a_n < L + 1 \leq M$. Therefore, for all values of $n$,the set $\{a_n: n \in \mathbb{N}\}$ is bounded above.\\[1ex]
            Similarly for the lower bound, take
            \[M \coloneqq \min\{a_1, a_2, \ldots, a_{n_0 -1}, L - 1\}\]
            If $n < n_0$, then it is in the set $\{a_1, a_2, \ldots, a_{n_0-1}\}$
            $M'$ is at most the minimum of this set, so $\forall n < n_0$, $a_n \geq M'$. If $n \geq n_0$, by the definition of the limit we know that for all $n \geq n_0$, $a_n > L - 1$. So $M'$ is at most $L - 1$. Therefore $\forall n \geq n_0$, $a_n > L - 1 \geq M'$. Therefore, the set is bounded below and above, so it is bounded.
   \end{proof}

   \begin{manualprop}{3.2.1}[Boundedness Test]
    Let $\infseq{a}$ be a sequence of real numbers. Suppose that
    \begin{enumerate}[label=(\roman*)]
        \item $a_n \geq 0$ for all $n$, and
        \item There is a bound $M \in \real$ on the partial sums, so that
        \[\sum_{n=1}^N a_n \leq M\]
        for all $N \in N_{\geq 1}$.
    \end{enumerate}
    Then $\infser{a}$ converges.
\end{manualprop}
\begin{proof}
    Since $a_n \geq 0$, the partial sums $(S_N)_{N=1}^\infty$ satisfy
    \[S_N \leq S_{N+1} \text{ for all $N$.}\]
    In other words, $(S_N)^\infty_{N=1}$ is an increasing sequence. The second condition ensures that the sequence is bounded above. Therefore, by the Monotone Convergence Criterion, it converges. Therefore, $\sum\limits_{n=1}^\infty a_n$ converges.
\end{proof}


\begin{manualprop}{4.2.3}
    Let $\infseq{a}$ be a sequence in $\real^d$, with 
    \[a_n = (a_n^{(1)}, \ldots, a_n^{(d)}) \text { for each } n \in \nat\]
    and let $L = (L_1, \ldots, L_d) \in \real^d$. Then
    \[\lim_{n\rightarrow\infty} a_n = L\]
    if and only if, for each $i = 1, \ldots, d$,
    \[\lim_{n\rightarrow\infty} a_n^{(i)} = L_i\]
\end{manualprop}
\begin{proof}
    ($\implies$) Assume that $\lim\limits_{n\rightarrow\infty} a_n = L$. Then, for each $i = 1, \ldots, d$, we have that $|x_i|^2 \leq \sum_{i=1}^{d} x_i^2 = ||x||_2^2$, therefore 
    \[|x_i| \leq ||x||_2\]
    Using this fact, we then have each component of $||a_n - L||_2$ is less than or equal to it. So 
    \[|a_n^{(i)} - L_i| \leq ||a_n - L||_2\]
    \[-||a_n - L||_2 \leq a_n^{(i)} - L_i \leq ||a_n-L||_2\]
    Since $\lim\limits_{n\rightarrow \infty}a_n = L$, we have 
    $\lim\limits_{n\rightarrow\infty} a_n - L = 0$. By the Squeeze theorem, it follows that 
    \[\lim_{n \rightarrow \infty} a_n^{(i)} - L_i = 0 \implies \lim_{n\rightarrow \infty} a_n^{(i)} = L\]
    ($\impliedby$) Suppose for each $i = 1, \ldots, d$, we have 
    \[\lim_{n\rightarrow\infty} a_n^{(i)} = L_i\]
    Then, from the definition of $||\cdot||_2$, we have 
    \[||a_n -L||_2^2 = (a_n^{(1)} - L_1)^2 + \cdots + (a_n^{(d)} - L_d)^2\]
    Now taking limits of both sides
    \[\lim_{n\rightarrow\infty} ||a_n - L||_2^2 = \lim_{n\rightarrow\infty} (a_n^{(1)} - L_1)^2 + \cdots + \lim_{n\rightarrow\infty} (a_n^{(d)} - L_d)^2\]
    Now we'll prove exercise 2.2.5 which states that if $\infseq{a}$ is a sequence of non-negative real number converging to $L \geq 0$, then $\lim\limits_{n \rightarrow \infty} \sqrt{a_n}$ converges to $\sqrt{L}$. To prove this we will consider two cases where $L = 0$, and $L > 0$. 
    \begin{itemize}
        \item \textbf{Case 1, L = 0}: Suppose $\infseq{a} \rightarrow 0$, then from the definition of convergence we have that $\forall \epsilon > 0$, $\exists n_0$ such that $\forall n \geq n_0$, 
        \[|a_n - 0| < \epsilon\]
        Since $\epsilon$ is abritrary, we'll replace $\epsilon$ with $\epsilon^2$, so 
        \[|a_n - 0| < \epsilon^2\]
        Then we get 
        \[|a_n - 0| = |a_n| < \epsilon^2 \implies \sqrt{|a_n|} < \epsilon\]
        Therefore, $\sqrt{a_n} \rightarrow 0$ by the definition of convergence.
        \item \textbf{Case 2, $L > 0$}: Suppose $\infseq{a} \rightarrow L > 0$. Let $\epsilon >0$ be given, then there exists $n_0$ such that for all $n \geq n_0$,
        \[|a_n - L| < \epsilon\]
        We much such that $|\sqrt{a_n} - \sqrt{L}| < \epsilon$
        \[|\sqrt{a_n} - \sqrt{L}| \cdot \frac{\sqrt{a_n} + \sqrt{L}}{\sqrt{a_n} + \sqrt{L}} = \frac{|(\sqrt{a_n} - \sqrt{L})(\sqrt{a_n} + \sqrt{L})|}{\sqrt{a_n} + \sqrt{L}}\]
        Since $\sqrt{a_n} + \sqrt{L}$ is positive because $a_n, L \geq 0$, then $\sqrt{a_n} + \sqrt{L} = |\sqrt{a_n} + \sqrt{L}|$, then using the fact that $|a|\cdot|b| = |a\cdot b|$, we get 
        \[\frac{|a_n - \sqrt{L}\sqrt{a_n} + \sqrt{L}\sqrt{a_n} + L|}{\sqrt{a_n} + \sqrt{L}} = \frac{|a_n - L|}{\sqrt{a_n} + \sqrt{L}} \leq \frac{|a_n - L|}{\sqrt{L}}\]
        Now if we replace $\epsilon$ with $\epsilon \over \sqrt{L}$, we get 
        \[|\sqrt{a_n} - \sqrt{L}| < \frac{|a_n - L|}{\sqrt{L}} < \frac{\epsilon}{\sqrt{L}} \implies |\sqrt{a_n} - \sqrt{L}| < \epsilon\]
        Therefore, $\lim\limits_{n\rightarrow\infty}\sqrt{a_n} = \sqrt{L}$
    \end{itemize}
    Now going back to the original proof, 
    \[\lim_{n\rightarrow \infty} ||a_n - L||_2^2 = 0\]
    So from exercise 2.2.5 we have
    \[\lim_{n\rightarrow \infty} \sqrt{||a_n - L||^2_2} = \sqrt{0}\]
    Therefore,
    \[\lim_{n \rightarrow \infty} ||a_n - L||_2 = 0\]
    as required.
\end{proof}

\begin{manualprop}{4.3.4}
    Given $a \in \real^d$, and $r > 0$, the open ball $B(a,r)$ is an open set. 
\end{manualprop}
\begin{center}
    \textbf{Note: This is example 4.3.4 from the professors notes.}
\end{center}
\begin{proof}
    Recall the definition of an open set is that for any $x$ in the set, we can define an open ball (or epsilon neighborhood) around $x$ such that the ball is contained in the set. So we want an open ball $B(x;\epsilon)$ such that $B(x;\epsilon) \subseteq B(a;r)$. To see this, let $x \in B(a;r)$, so that $||x-a||_2 < r$. Define 
    \[\epsilon \coloneqq r - ||a-x||_2 > 0\] 
    Now take some element $y \in B(x;\epsilon)$, then we want to show that tis element is contained in $B(a;r)$. So, $y \in B(x;\epsilon)$, so that $||y-x||_2 < \epsilon$. Then,
    \begin{align*}
        ||y-a||_2 = ||y - x + x -a||_2 &\leq ||y - x||_2 + ||x -a||_2\\
        &< \epsilon + ||a-x||_2 = r
    \end{align*}
    Therefore, 
    \[||y-a||_2 < r\]
    So $y \in B(a;r)$ as required, so $B(a;r)$ is an open set.
\end{proof}

\begin{manualprop}{4.3.9}
    Let $F \subseteq \real^d$. Then $F$ is closed if and only if for every sequence $\infseq{a}$ in $F$, if $\infseq{a}$ converges then 
    \[\lim_{n\rightarrow\infty} a_n \in F\]
\end{manualprop}
\begin{proof}
    ($\implies$) Assume that $F$ is closes, so $\real^d \setminus F$ is open. Let $\infseq{a}$ be a sequence in $F$ which converges. Suppose for a contradiction that $L \coloneqq \lim\limits_{n\rightarrow\infty} a_n$ is not in $F$. So $L \in \real^d \setminus F$. Then by openness, there exists $\epsilon > 0$ such that $B(L; \epsilon) \subseteq \real^d \setminus F$. Using the definition of convergence, we have that there exists $n$ such that 
    \[||a_n - L||_2 < \epsilon\]
    But, this means that 
    \[a_n \in B(L;\epsilon) \subseteq \real^d \setminus F\]
    which contradicts that $a_n \in F$. Therefore, $L \in F$ as required.\\[1ex]
    ($\impliedby$) Assume that every sequence in $F$ that converges, the limit is in $F$. We want to show that $\real^d \setminus F$ is open. Take a point $x \in \real^d \setminus F$, and suppose for a contradiction that there is no $\epsilon > 0$ such that $B(x; \epsilon) \subseteq \real^d \setminus F$. We can take $\epsilon \coloneqq \frac{1}{n}$. So for each $n \in \nat_{\geq 1}$, $B(x;\frac{1}{n})$ is not contained in $\real^d \setminus F$. So, there must be a point in $B(x;\frac{1}{n})$ that is in $F$. Let $a_n$ be such a point. Then since it is in $B(x;\frac{1}{n})$, we have that
    \[||a_n-x||_2 < \frac{1}{n}\]
    $x \in \real^d \setminus F$ so $x \neq a_n$, so 
    \[0 < ||a_n-x||_2 < \frac{1}{n}\]
    So by the squeeze theorem,
    \[\lim_{n\rightarrow\infty} ||a_n-x||_2 = 0\]
    So, 
    \[\lim_{n \rightarrow \infty} a_n = x\]
    By our hypothesis, every sequence which converges, the limit is in $F$. So $x \in F$, but by our assumption $x \in \real^d \setminus F$, which is a contradiction. Therefore, $\real^d \setminus F$ is open as required.
\end{proof}

\begin{manualprop}{5.2.5}
    Let $X \subseteq \real^d$ and let $Y \subseteq \real^m$. Let $f : X \mapsto Y$ and $g : Y \mapsto \real^n$ be functions. Let $a \in X$. Suppose that $f$ is continuous at $a$ and $g$ is continuous at $f(a)$. Then $g \circ f$ is continuous at $a$.
\end{manualprop}
\begin{proof}
    Let $\epsilon > 0$ be given. Since $g$ is continuous at $f(a)$, there exists $\eta > 0$ such that for all $y \in Y$, if $||y - f(a)||_2 < \eta$, then $||g(y) - g(f(a))||_2 < \epsilon$ (by the definition of continuity). Since $f$ is continuous at $a$, we can find $\delta$ such that for all $x \in X$, if $||x-a||_2 < \delta$, then $||f(x) - f(a)||_2 < \eta$. Now since these inequalities hold for all $x \in X$ and $y \in Y$. We can take $y = f(x)$. So if $x \in X$ and $||x-a||_2 < \delta$, then $||f(x) - f(a)||_2 < \eta$, and so 
    \[||g(f(x)) - g(f(a))||_2 < \epsilon\]
    as required.
\end{proof}
    
\begin{manualtheorem}{5.3.2}
    Let $K \subseteq \real^d$ be compact and let $f: K \mapsto \real^m$ be a continuous function. Then its image, $f(K)$ is also compact. 
\end{manualtheorem}
\begin{proof}
    Let $\infseq{y}$ be a sequence in $f(K)$, we need to find a subsequence that converges to a point in $f(K)$. Recall that $f(K) = \{f(k): k \in K\}$. So, we can find $x_n \in K$ for each $n$ such that $y_n = f(x_n)$. Since $K$ is compact, there is a subsequence $(x_{n_k})_{n=1}^\infty$ which converges to a point $a \in K$. $f$ is continuous so 
    \[\lim_{x\rightarrow a} f(x) = f(a)\]
    therefore by the Sequential Characterization of Limits, $\lim_{k \rightarrow \infty} f(x_{n_k} ) = f(a)$. So, the subsequence $(y_{n_k})_{n=1}^\infty$ converges to $f(a) \in f(K)$ as required.
\end{proof}

\begin{manualcorollary}{5.3.3}[Extreme Value Theorem]
    Let $K \subset \real^d$ be compact and nonempty, and let $f: K \mapsto \real$ be a continuous function. Then there exists $x_{\min}, x_{\max} \in K$ such that for all $x \in K$, 
    \[f(x_{\min}) \leq f(x) \leq f(x_{\max})\]
    In otherwords, the image of $f$ is bounbded above and below, and it attains its boundes. 
\end{manualcorollary}
\begin{proof}
    By the previous theorem, $f(K)$ is compact. By the Heine-Borel theorem, $f(K)$ is closed and bounded. Since it is bounded, by completeness we have $\sup f(K)$ and $\inf f(K)$ exist. We can construct a sequence $(a_n)_{n=1}^\infty$ which converges to $\sup f(K)$. Let $L \coloneqq \sup f(K)$. For any given $\epsilon > 0$, $L - \epsilon < L$ so it is not an upperbound. Therefore we have some element $a \in f(K)$ such that $L - \epsilon < a \leq L$. Set $\epsilon \coloneqq \frac{1}{n}$ for $n \in \nat_{\geq 1}$. So we can define the seqeunce members of $(a_n)_{n=1}^\infty$ as 
    \[L - 1 < a_1 \leq L\]
    \[L - \frac{1}{2} < a_2 \leq L\]
    \[L - \frac{1}{3} < a_3 \leq L\]
    \[L - \frac{1}{n} < a_n \leq L\]
    This holds for all $n \in \nat_{\geq1}$ since if $L - \frac{1}{n}$ is an upper bound, then it contradicts that $L = \sup f(K)$. Then by the Squeeze theorem, we have that this sequence converges to $L$. So since $f(K)$ is closed, $L \in f(K)$. Similarly, we can construct a sequence that converges to $\inf f(K)$. Therefore, by boundedness we have 
    \[\inf f(K) \leq f(K) \leq \sup f(K) \]
    Since $f(K)$ is closed, there exists $x_{\min} \in K$ such that $f(x_{\min}) = \inf f(K)$. Similarly, there exists $x_{\max} \in K$ such that $f(x_{\max}) = \sup f(K)$. 
    \begin{center}
        \textbf{Note:} It follows that $\sup f(K) \in f(K)$ because $f(K)$ is closed, but I wanted to prove it in an intuitive way to be thorough.
    \end{center}
\end{proof}

\begin{manualtheorem}{5.4.4}
    Let $I \subseteq \real$ be an interval and let $f : I \mapsto \real$ be an injective continuous functiion. Then $f^{-1}: f(I) \mapsto \real$ is continuous. 
\end{manualtheorem}
\begin{proof}
    From Lemma 5.4.3, since $f$ is injective then $f$ is either strictly increasing or strictly decreasing. Assume without loss of generality that $f$ is strictly increasing. Let $b \in f(I)$, we want to show that $\lim\limits_{y\rightarrow b} f^{-1}(y) = f^{-1}(b)$. Let $\epsilon > 0$ be given. Set $a \coloneqq f^{-1}(b)$, so $f(a) = b$. Consider 2 cases where $a \in I^{\circ}$ and $a \in \partial I$. When $a \in \partial I$ the argument follows very similarly so we will suppose that $a \in I^{\circ}$. Since $\epsilon$ is arbitrary, we can assume $a - \epsilon, a + \epsilon \in I$. Then we have
    \[f(a - \epsilon) < f(a) < f(a + \epsilon)\]
    Then set 
    \[\delta \coloneqq \min\{f(a) - f(a - \epsilon), f(a + \epsilon) - f(a)\} > 0\]
    We want to prove that if $y \in f(I)$ and $|y - b| < \delta$, then $|f^{-1}(y) - f^{-1}(b)| < \epsilon$. Suppose for a contradiction that this is false, then 
    \[f^{-1}(y) \leq f^{-1}(b) \text{ or } f^{-1}(y) \geq f^{-1}(b) + \epsilon\]
    Set $x \coloneqq f^{-1}(y)$ so $f(x) = y$. So we have 
    \[x \leq a - \epsilon \text{ or } x \geq a + \epsilon\]
    In the first case since $f$ is strictly increasing we get 
    \[y = f(x) \leq f(a - \epsilon) \leq f(a) - \delta = b - \delta\]
    So $y = b - \delta$. In the second case we get 
    \[y = f(x) \geq f(a + \epsilon) \geq f(a) + \delta = \beta + \delta\]
    In both cases, this contradicts that $|y - b| < \delta$ so we must have that our claim was true. 
\end{proof}

\begin{manualprop}{6.1.5}
    Let $X \subseteq \real$, let $f: X \mapsto \real$ be a function, let $a \in X$ be a non-isolated point. If $f$ is differentiable at $a$ then $f$ is continuous at $a$.
\end{manualprop}
\begin{proof}
    We want to show that $\lim\limits_{x\rightarrow a}f(x) - f(a)$. We have 
    \begin{align*}
        \lim_{x\rightarrow a}f(x) - f(a) &= \lim_{x\rightarrow a}\left(\frac{f(x) - f(a)}{x - a}(x - a)\right)\\
        &= \left(\lim_{x\rightarrow a} \frac{f(x) - f(a)}{x-a}\right)\left(\lim_{x\rightarrow a} x- a\right)\\
        &= f'(a) \cdot 0 = 0 \implies \lim_{x\rightarrow a} f(x) = f(a)
    \end{align*}
    We can use algebra of limits since both limits are well defined.
\end{proof}

\begin{manualtheorem}{6.3.2}
    Let $X \subseteq \real$, let $f: X \mapsto \real$ and let $a \in X$ be an interior point. If $f$ has a local maximum or local minimum at $a$ and $f$ is differentiable at $a$, then $f'(a) = 0$.
\end{manualtheorem}
\begin{proof}
    Assume that $f$ is a local minimum at $a$. Let $r > 0$ be such that $(a-r, a+r) \subseteq X$ and 
    \[f(a) \leq f(x) \ \ \forall x \in (a - r, a + r)\]
    In other words, $f(a)$ is the local minimum in the $r$ neighborhood of $a$. Since $f$ is differentiable, the limit 
    \[\lim_{x\rightarrow a} \frac{f(x) - f(a)}{x-a}\]
    exists. By the Sequential Characterization of Limits, for any sequence $(x_n)_{n=1}^\infty$ in $X$ converging to $a$, we have 
    \[f'(a) = \lim_{n\rightarrow\infty} \frac{f(x_n) - f(a)}{x_n - a}\]
    We want to construct two sequences that converge to $a$ from the left and right, so from the right consider the sequence with members $x_n \in (a, a+r)$
    \[x_n \coloneqq a + \frac{r}{n+1}\]
    Then $x_n - a > 0$ and $f(x_n) - f(a) \leq 0$ since $f(a) \leq f(x)$ $\forall x \in (a - r, a + r)$. So this implies that 
    \[f'(a) = \lim_{n\rightarrow \infty} \frac{f(x_n) = f(a)}{x_n-a} \geq 0\]
    Similarily, we can construct a sequence from the left with members $x_n \in (a-r, a)$, take 
    \[x_n \coloneqq a - \frac{r}{n+1}\]
    So $x_n - a < 0$, thus
    \[f'(a) = \lim_{n \rightarrow \infty} \frac{f(x_n)-f(a)}{x_n-a} \leq 0\]
    Then by combining these inequalities, we get that $f'(a) = 0$, as required.
\end{proof}

\begin{manualtheorem}{6.4.1}[Rolle's Theorem]
    Let $f: [a,b] \mapsto \real$ be a continuous function that is differentiable on $(a,b)$. If $f(a) = f(b)$, then there exists $x_0 \in (a,b)$ such that
    \[f'(x_0) = 0\]
\end{manualtheorem}
\begin{proof}
    If $f$ is constant then it follows that $f'(x_0) = 0$ for all $x_0 \in (a,b)$. Otherwise, by the Extreme Value theorem there exists $x_{\min}, x_{\max}$ such that 
    \[f(x_{\min}) \leq f(x) \leq f(x_{\max})\]
    Since $f$ is not constant, we know that we have either (or both)
    \[f(x_{\min}) < f(a) = f(b) \text{ or } f(x_{\max}) > f(a) = f(b)\]
\end{proof}

\begin{manualtheorem}{7.1.11}
    If $f: [a,b] \mapsto \real$ is continuous, then $f$ is integrable. 
\end{manualtheorem}

\begin{proof}
    Since $f$ is continuous, then by definition there exists $\delta$ such that if $x,y \in [a,b]$ and $|x - y| < \delta$, then $|f(x) - f(y)| < \epsilon$. By proposition 5.3.6, we have that $f([a,b]) = [c,d]$ for some $c,d \in \real$. Let $P = \{t_0, \ldots, t_{n-1}\}$ be a partion such that 
    \[|t_i - t_{i-1}| < \delta\]
    for all $i$. Since $f$ is uniformly continuous, then for $x,y \in [t_{i-1}, t_i]$ if $|x - y| < \delta$, then $|f(x) - f(y)|$. So, the maximum distance from $|f(x) - f(y)|$ cannot exceed $\epsilon$. In otherwords, 
    \[M_i(P,f) - m_i(P,f) < \epsilon\]
     By the extreme value theorem, we can find a maximum and minimum $f(x_{\min})$, $f(x_{\max})$ for any interval $[t_{i-1}, t_i]$. From the partitions we have that 
    \[f(x_{\min}) = m_i(P,f) \text{ and } f(x_{\max}) = M_i(P,f)\]
    Now if we take the sums of these intervals, we get
    \begin{align*}
        U(P,f) - L(P,f) &= \sum_{i=1}^n (M_i(P,f) - m_i(P,f))(t_i - t_{i-1})\\
        &< \sum_{i=1}^n \epsilon(t_i - t_{i-1})\\
        &= \epsilon \sum_{i=1}^n (t_i - t_{i-1})\\
        &= \epsilon (t_n-t_0)\\
    \end{align*}
    Since $\epsilon$, is arbitrary, we can replace $\epsilon$ with $\frac{\epsilon}{b-a}$. Then we get that
    \[U(P,f) - L(P,f) < \frac{\epsilon}{b-a} (t_n-t_0) = \epsilon\]
    Then from proposition 7.1.10, which states that $f$ is integrable if and only if for every $\epsilon > $, there exists a partition $P$ such that 
    \[U(P,f) - L(P,f) < \epsilon\]
    we have that $f$ is integrable.
\end{proof}
\begin{manualprop}{}
    Let $X$ be a set, let $(f_n: X \mapsto \real^m)$ be a sequence of functions, and let $f: X \mapsto \real^m$. If $f_n$ converges uniformly to $f$, then $f$ converges pointwise to $f$.
\end{manualprop}
\begin{proof}
    Suppose $f_n$ converge uniformaly to $f$, then $\forall \epsilon > 0$, $\exists n_0$ such that $\forall n \geq n_0$, 
    \[||f_n(x) - f(x)||_2 <\epsilon \ \ \forall x \in X\]
    Therefore, from the definition of the limit, we have that 
    \[\lim_{n\rightarrow \infty} f_n(x) = f(x) \ \ \forall x \in X\]
    as required. 
\end{proof}

\begin{manualtheorem}{8.2.6}[The Weierstrass $M$-test]
    Let $X$ be a set, let $(f_n: X \rightarrow \real)_{n=1}^\infty$ be a sequence of functions, and let $(M_n)_{n=1}^\infty$ be a sequence of non-negative real numbers. Suppose that the follow hold
    \begin{enumerate}[label=(\roman*)]
        \item $|f_n(x)| \leq M_n$ for all $x \in X$, and 
        \item $\sum\limits_{n=1}^\infty M_n$ converges 
    \end{enumerate}
    Then $\sum\limits_{n=1}^\infty f_n$ converges uniformly. 
\end{manualtheorem}

\begin{proof}
    Consider the sequence of partial sums 
    \[s_n = \sum_{m=1}^n f_m\]
    Take $N, M \in \nat$ such that $N \leq M$. Then
    \begin{align*}
        |s_{M}(x) - s_{N}(x)| &= \left|\sum_{m = N+1}^{M} f_m(x)\right|\\
        &\leq \sum_{m = N+1}^{M} |f_m(x)|\\
        &\leq \sum_{m=N+1}^{M} M_m\\
    \end{align*}
    Then we have that $\sum\limits_{n=1}^\infty M_n$ converges. So for any $\epsilon > 0$, we can find $N_0$ such that for any $M \geq N \geq N_0$, we have 
    \[|s_M (x) - s_N(x)| \leq \sum_{m = N+1}^M M_m < \epsilon\]
    Hence, the sequence $s_n$ converges uniformly to $\sum\limits_{n=1}^\infty f_n$.
\end{proof}

\begin{manualtheorem}{8.3.1}
    Let $X \subseteq \real^d$ be a set and $a \in X$. Let $(f_n: X \rightarrow \real^m)_{n=1}^\infty$ be a sequence of functions which converges unifromly to $f: X \rightarrow \real^m$. If each $f_n$ is continuous at $a$, then so is $f$. Hence, if each $f_n$ is continuous, then so is $f$.
\end{manualtheorem}
\begin{proof}
    Since $f_n$ converges uniformly to $f$, then for any $\epsilon > 0$, there exists $n_0$ such that 
    \[||f_n(x) - f(x)||_2 < \epsilon\]
    Now take any $n \geq n_0$, we want to show that there exists $\delta > 0$ such that if 
    \[||x-a||_2 < \delta\]
    then 
    \[||f(x) - f(a)|| < \epsilon\]
    Since $f_n$ is continuous, then for all $x \in X$, we have that when $||x-a||_2 < \delta$, then
    \[||f_n(x) - f_n(a)||_2 < \epsilon\]
    Now take $x \in X$ such that $||x-a||_2 < \delta$, then
    \begin{align*}
        ||f(x) - f(a)||_2 &= ||f(x) - f_n(x) + f_n(x) - f_n(a) + f_n(a) - f(a)||_2\\
        &\leq ||f(x) - f_n(x)||_2 + ||f_n(x) - f_n(a)||_2 + ||f_n(a) - f(a)||_2\\
    \end{align*}
    Then since $f_n$ converges uniformly to $f$, we have that 
    \[||f(x) - f_n(x)||_2 < \epsilon \text{ and } ||f_n(a) - f(a)||_2 < \epsilon\]
    Therefore,
    \[||f(x) - f_n(x)||_2 + ||f_n(x) - f_n(a)||_2 + ||f_n(a) - f(a)||_2 = \epsilon + \epsilon + \epsilon\]
    Since $\epsilon$ is arbitrary, we can replace it with $\frac{\epsilon}{3}$, and we get 
    \[||f(x) - f(a)||_2 < \frac{\epsilon}{3} + \frac{\epsilon}{3} + \frac{\epsilon}{3} = \epsilon\]
    as required.
\end{proof}

\begin{manualtheorem}{9.1.15}
    Let $\sum\limits_{n=0}^\infty a_n(x-c)^n$ be a power series, and define 
    \[R \coloneqq \frac{1}{\limsup\limits_{n\rightarrow\infty} \sqrt[n]{|a_n|}}\]
    (this value is 0 when $\limsup$ is $\infty$ and $\infty$ when $\limsup$ is 0). Then for $b \in \real$, 
    \begin{enumerate}[label=(\roman*)]
        \item if $|b-c|< R$, then$\sum\limits_{n=0}^\infty a_n(x-c)^n$ converges, while 
        \item if $|b-c| \geq R$, then $\sum\limits_{n=0}^\infty a_n(x-c)^n$ diverges.
    \end{enumerate}
\end{manualtheorem}
\begin{proof}
    Let $b \in \real$ with $b \neq 0$, since the case where $b = c$ is trivial. We have 
    \[\limsup_{n\rightarrow \infty} \sqrt[n]{|a_n(b-c)^n|} = \limsup_{n\rightarrow \infty}|b-c|\sqrt[n]{|a_n|} = |b-c|\limsup_{n\rightarrow \infty}\sqrt[n]{|a_n|}= \frac{|b-c|}{R}\]
    Hence by the root test, if $\frac{|b-c|}{R} < 1$ (i.e $|b-c| < R$), the series converges, while if $\frac{|b-c|}{R} \geq 1$ $|b-c|>R$, the series diverges as required 
\end{proof}

\section*{Pointwise vs Uniform Convergence for Sequences of Functions}
For pointwise convergence, the definition is as follows: Let $X$ be a set and $(f_n: X \rightarrow \real^m)^\infty_{n=1}$ be a sequence of functions, and let $f: X \rightarrow \real^m$. Then $f_n$ converges to $f$ \emph{pointwise} if for all $x \in X$, we have that
\[\lim_{n\rightarrow\infty} f_n(x) = f(x)\]
Now the $\epsilon$-$N$ characeterization for pointwise convergence is, for any $\epsilon > 0$ and $\forall x \in X$, there exists $N \in \nat$ such that for all $n \geq N$,\
\[||f_n(x) - f(x)||_2 < \epsilon\]
Now for uniform convergence, we have that for any $\epsilon > 0$, there exists $N \in \nat$ such that for all $n \geq N$,
\[||f_n(x) - f(x)||_2 < \epsilon \ \ \forall x \in X\]
The key difference between these 2 definitions is the order of the quantifiers. For pointwise convergence, we have that for all $\epsilon > 0$, and $\forall x \in X$, there exists $N \in \nat$. But for uniform convergence, its for all $\epsilon > 0$, there exists $N \in \nat$. So, in point wise converge $N$ is "brought into existence" after the $x$ is, so $N$ can depend on $X$, whereas for uniform convergence, $N$ is "brought into existence" before the $x$ is, so $N$ \emph{cannot} depend on $X$ and must hold $\forall x$
%-------------- Lecture 1 ------------------------
\chapter{The Real Numbers $\real$}
\begin{center}
    \textbf{Summary:} $\real$ is a complete ordered field.
\end{center}
\section{Fields}
\begin{definition}
    A \emph{field} is a set F together with operations $+,\cdot$ satisfying
    \begin{itemize}
        \item (F1) $a+b=b+a$ $\forall a,b \in F$ (Commutativity)
        \item (F2) $(a + b) + c = a + (b + c)$ $\forall a,b,c \in F$     (Associativity)
        \item (F3) $\exists 0 \in F$ s.t $0 + a = a$ $\forall a \in F$ (Additive Identity)
        \item (F4) $\exists -a \in F$ s.t $a + (-a) = 0$ $\forall a \in F$ (Additive Inverse)
        \item (F5) $a\cdot b = b \cdot a$ $\forall a,b \in F$ (Commutativity)
        \item (F6) $(a \cdot b) \cdot c = a \cdot (b \cdot c)$ $\forall a,b,c \in F$ (Associativity)
        \item (F7) $\exists 1 \in F$ s.t $1 \cdot a = a$ $\forall a \in F$ (Multiplicative Identity)
        \item (F8) $\forall a \in F \setminus \{0\}$ $\exists a^{-1} \in F$ s.t $a^{-1} \cdot a = 1$ (Multiplicative Inverse)
        \item (F9) $a \cdot (b + c) = a \cdot b + a \cdot c$ $\forall a,b,c \in F$ (Distributivity)
    \end{itemize}
\end{definition}

\section{Ordered Fields}

\begin{definition}
    An \emph{ordered field} is a field $F$ along with a relation $<$ satistfying
    \begin{itemize}
        \item (O1) $\forall a,b,c \in F$, if  $a < b$ and $b < c$ then $a<c$ (Transitivity)
        \item (O2) $\forall a,b \in F$ exactly one of the following is true,
        \[a < b \text{ or } a = b \text{ or } b < a\]
        \item (O3) $\forall a,b,c \in F$, if $a < b$, then $a + c < b + c$
        \item $\forall a,b,c \in F$, If $a < b$ and $0 < c$, then $ac < bc$
    \end{itemize}
\end{definition}

\section{Complete Ordered Fields}
\begin{definition}
    Let F be an ordered field. Let $S \subseteq F$. An upper bound or $S$ is some $M \in F$ s.t $\forall x \in S$
    \[x \leq M\]

\end{definition}



\chapter{Completeness of $\real$, Absolute Value, Sequences }
TBC.
\chapter{Convergence of Sequences}
TBC. 
\chapter{Properties of Convergence, Squeeze Theorem, Monotone Sequences}
TBC. 
\chapter{Subsequences, Cauchy Sequences}
TBC.
\chapter{Limsup and Liminf}
TBC.

%----------------------------Lecture 7 ----------------------------------
\chapter{Series}
\textbf{Recall:}
\[\infser{a} = \lim_{N\rightarrow \infty} \sum_{n=1}^N a_n\]
$\infser{a}$ "diverges" if above limit does not exisit.
\begin{prop}
    Suppose $\infseq{a}, \infseq{b}$ converges. Then
    \begin{enumerate}[label=(\roman*)]
        \item 
        \[\sum_{n=1}^\infty a_n + b_n = \infser{a} + \infser{b}\]
        \item
        \[\sum_{n=1}^\infty cb_n = c\infser{b} \ \forall c \in \real\]
    \end{enumerate}
    This says 
    \[V \coloneqq \{\infseq{a}: \infser{a} \ converges\}\]
    is a vector space over $\real$.
\end{prop}

\begin{center}
    \textbf{Note:}
    \[\left(\sum_{n=1}^N a_n\right) \left(\sum_{n=1}^N b_n\right) \neq \sum_{n=1}^\infty a_nb_n\]
\end{center}

\begin{proof}
    \textbf{Exercise.}
\end{proof}

\begin{prop}
    \begin{center}
        $\infser{a}$ converges $iff$ $\sum_{n=k}^\infty a_k$ converges.
    \end{center}
\end{prop}
\begin{proof}
    \textbf{Exercise.}
\end{proof}

\textbf{Example:} TBC.

\begin{prop}
    If $\infser{a}$ converges, then $a_n \rightarrow 0$.
\end{prop}


\section{Divergence Test}

\begin{prop}[Divergence Test]
    If $a_n\not\rightarrow 0$, then $\infser{a}$ diverges.
\end{prop}
\begin{proof}
    $\sum_{n=1}^\infty a_n$ converges $\implies S_n \rightarrow L$ for some $L$, where 
    \[S_n \coloneqq \sum_{n=1}^N a_n\]
    So, 
    \[a_n = S_n - S_{n-1}\]
    By the Algebra of Limits, 
    \begin{align*}
        \lim_{n\rightarrow \infty} a_n &= \lim_{n\rightarrow \infty} S_n - \lim_{n\rightarrow \infty} S_{n-1}\\
        &= L - L = 0
    \end{align*}
\end{proof}
\textbf{Example:} TBC.

\section{Convergence Tests}
\begin{prop}[Boundedness Test]
    Let $\infseq{a}$ bea  sequence, if 
    \begin{enumerate}[label=(\roman*)]
        \item $a_n \geq 0$
        \item There is an upper bound on the parital sums
        \[\exists M > 0 \ s.t \ \sum_{n=1}^N a_n \leq M\]
    \end{enumerate}
    Then $\infser{a}$ converges.
\end{prop}
\begin{proof}
    Let 
    \[S_N \coloneqq \sum_{n=1}^N a_n\]
    Then 
    \begin{align*}
        S_{N+1} &= S_N + a_{N+1}\\ 
        &\geq S_n
    \end{align*}
    So by the Monotone Convergence Criterion, $(S_N)_{N=1}^\infty$ converges $\iff$ it is bounded avove. By ($ii$), it is bounded.
\end{proof}

\begin{prop}[Comparison Test]
    TBC. 
\end{prop}

%----------------------------Lecture 8 ----------------------------------
\chapter{Ratio, Root, Alternating Series, and Integral Test, Cauchy Convergence, Topology of $\real^d$}
\section{Ratio Test}
\begin{prop}[Ratio Test]
    Let $\infseq{a}$ be a sequence of nonzero elements.
    \begin{enumerate}[label=(\roman*)]
        \item If 
        \[\limsup_{n \rightarrow \infty} \left| \frac{a_{n+1}}{a_n} \right|< 1\]
        then $\infser{a}$ converges absolutely.
        \item If 
        \[\liminf_{n\rightarrow \infty} \left| \frac{a_{n+1}}{a_n} > 1 \right| > 1\]
        then $\infser{a}$ diverges.
    \end{enumerate}
\end{prop}

\begin{proof}
    ~\newline
    \begin{enumerate}[label=(\roman*)]
        \item Let 
        \[q = \limsup_{n\rightarrow \infty} \left| \frac{a_{n+1}}{a_n} \right|\]
        where $q < 1$, $\exists r \in (q,1)$. By the definition of $\limsup$, 
        \[\left| \frac{a_{n+1}}{a_n} \right| \leq r \ \forall n \geq n_0\]
        for some $n_0 \in \nat$.
        \begin{align*}
            \left| \frac{a_{n_0+1}}{a_{n_0}} \right| &\leq r\\
            \left| a_{n_0+1} \right| &\leq r|a_{n_0}|\\
            \left| a_{n_0+2} \right| &\leq r|a_{n_0+1}| \leq r^2|a_{n_0}|\\
        \end{align*}
        By induction, we have
        \[0 \leq |a_{n_0 + k} \leq r^k |a_{n_0}|\]
        By the comparison test, 
        \[\sum_{k=1}^\infty |a_{n_0+ k}|\]
        converges since 
        \[\sum_{k=1}^\infty r^k|a_{n_0}|\]
        is a geometric sequence and $0 \leq r < 1$. 
        \[\sum_{n=1}^\infty |a_n|\]
        Converges, thus $\infser{a}$ converges absolutely.
        \item Let 
        \[q = \liminf_{n \rightarrow \infty} \left| \frac{a_{n+1}}{a_n}\right| > 1\]
        so $\exists n_0$ such that 
        \[\left| \frac{a_{n+1}}{a_{n}} \right| > 1\]
        for all $n \geq n_0$. Then
        \[|a_{n_0 + k} \geq |a_{n_0}| \ \forall k \geq 0 \]
        So $a_n \not\rightarrow 0$ as $n \rightarrow 0$, thus by the divergence test. $\infser{a}$ diverges.
    \end{enumerate}
\end{proof}

\begin{center}
    \red{\textbf{Note: The ratio test does not tell us anything when the limit is 1.}}
\end{center}
\textbf{For example:} 
\[\sum_{n\geq 1} \frac{1}{n}\]
diverges, but 
\[\frac{\frac{1}{n+1}}{\frac{1}{n}} = \frac{n}{n+1} \rightarrow 1\]
But on the other hand, 
\[\sum_{n\geq1} \frac{1}{n+1}\]
converges, and 
\[\frac{\frac{1}{n+1(n+2)}}{\frac{1}{n(n+1)}} = \frac{n+1}{n+2} \rightarrow 1\]
\section{Root Test}
\begin{prop}[Root Test]
    Let $\infseq{a}$ be a sequence of real numbers.
    \begin{enumerate}[label=(\roman*)]
        \item If 
        \[\limsup_{n\rightarrow \infty} \sqrt[n]{|a_n|} < 1\]
        then $\infser{a}$ converges absolutely.
        \item If 
        \[\limsup_{n\rightarrow \infty} \sqrt[n]{|a_n|} > 1\]
        then $\infser{a}$ diverges.
    \end{enumerate}
\end{prop}

\begin{proof}
    ~\newline
    \begin{enumerate}[label=(\roman*)]
        \item Let 
        \[q = \limsup_{n\rightarrow \infty} \sqrt[n]{|a_n|} < 1\]
        Then there exists $r \in (q,1)$ such that $\exists n_0$ 
        \[\sqrt[n]{|a_n|} \leq r\]
        for all $n \geq n_0$. Then,
        \[0 \leq |a_n| \leq r^n\]
        Therefore $\sum_{n=n_0}^\infty r^n$ converges since $0 < r < 1$. Then by the comparison test, $\sum_{n=n_0}^\infty |a_n|$ converges.
        \item Let
        \[q = \limsup_{n\rightarrow \infty} \sqrt[n]{|a_n|} > 1\]
        From exercise 2.7.3, there are infinitely many 
        \[\sqrt[n]{|a_n|} \geq 1 \implies |a_n| \geq 1\]
        Thus $a_n \not\rightarrow 0$ as $n \rightarrow \infty$, thus by the divergence test, $\infser{a}$ diverges.
    \end{enumerate}
\end{proof}

\textbf{Examples:}
\begin{itemize}
    \item 
    \[\sum_{n=1}^\infty \left(\frac{1}{2}\right)^n\]
    \[\sqrt[n]{\left(\frac{1}{2}\right)^2} = \frac{1}{2}\]
    \item
    \[\infser{\left(\frac{n}{2n+1}\right)}\]
    \[\sqrt[n]{\left(\frac{n}{2n+1}\right)^n} = \frac{1}{2^{n+1}} \rightarrow \frac{1}{2}\]
    \item 
    \[\infser{\frac{1}{n}}\]
    \[\sqrt[n]{\frac{1}{n}} = \frac{1}{n^{\frac{1}{n}}} = \frac{1}{e^{\frac{\ln n}{n}}} \rightarrow 0\]
\end{itemize}
\section{Alternating Series Test}
\begin{prop}[Alternating Series Test]
    Let $\infseq{a}$ be a sequence of real numbers. Suppose
    \begin{enumerate}[label=(\roman*)]
        \item $\infseq{a}$ is decreasing
        \item $\lim_{n\rightarrow \infty} a_n = 0$
    \end{enumerate}
    Then 
    \[\sum_{n=1}^\infty (-1)^{n+1}a_n\]
    converges. Moreover, for any $N$, 
    \[\sum_{n=1}^{2N} (-1)^{n+1}a_n \leq \sum_{n=1}^\infty (-1)^{n+1}a_n \leq \sum_{n=1}^{2N-1} (-1)^{n+1}a_n\]
\end{prop}
\begin{proof}
    $a_n \geq 0$ $\forall n$, since $a_n \rightarrow 0$ as $n \rightarrow \infty$, and $\infseq{a}$ is decreasing. Let
    \[S_N \coloneqq \sum_{n=1}^N (-1)^{n+1}a_n\]
    If $N$ is even, 
    \[S_{N+2} = S_N + a_{N+1} - a_{N+2} \geq S_N\]
    Since 
    \[a_{N+2} \leq a_{N+1}\]
    So $(S_{2N})_{N=1}^\infty$ is an increasing sequence and $(S_{2N -1})_{N=1}^\infty$ is a decreasing sequence. So by the monotone convergence criterion, both sequence converge. 
    \[S_{2N - 1} + a_{2N} = S_{2N}\]
    \[a_{2N} \xrightarrow{N \rightarrow \infty} 0 \]
    So, 
    \[\lim_{n\rightarrow \infty} S_{2N-1} = \lim_{N \rightarrow \infty} S_{2N} = L \implies \lim_{N\rightarrow \infty} S_N = L\]
    $(S_{2N})^\infty_{N=1}$ is increasing, so 
    \[L = \sup \{(S_{2N})_{N=1}^\infty\} \implies S_{2N} \leq L\]
    Similarily, $(S_{2N-1})^\infty_{N=1}$ is decreasing, so
    \[L = \inf\{(S_{2N_1})_{n=1}^\infty\} \implies S_{2N - 1} \geq L\] 

\end{proof}

\section{Integral Test}

\begin{prop}
    Let $f: [1,\infty) \rightarrow \real$. Suppose that
    \begin{enumerate}[label=(\roman*)]
        \item $f(x) \geq 0$ $\forall x \in [1,\infty)$
        \item $f$ is decreasing
    \end{enumerate}
    Then 
    \begin{center}
        $\sum_{n=1}^\infty f(n)$ converges $\iff$ the improper integral $\int_1^\infty f(x) \ dx$ converges.
    \end{center}
\end{prop}

\section{Cauchy Convergence Criterion for Series}

\begin{prop}
    Let $\infseq{a}$ be a sequence of real numbers. Then
    \begin{center}
        $\sum_{n=1}^\infty a_n$ converges $\iff$ $\forall \epsilon > 0$ $\exists N_0$ such that $N \leq M \leq N_0$, $|\sum_{n=M}^N a_n| \epsilon$. 
    \end{center}
\end{prop}
\begin{proof}
    Let 
    \[S_N \coloneqq \sum_{n=1}^N a_n\]
    Then $\sum_{n=1}^\infty a_n$ converges $\iff$ $(S_N)_{N=1}^\infty$ converges. Cauchy convergence criterion for sequences says that 
    \begin{center}
        $(S_N)^\infty_{n=1}$ converges $\iff$ it is cauchy.
    \end{center}
    \[\iff \forall \epsilon > 0 \ \exists N_0 \ s,t \ |S_N - S_M| < \epsilon \ \forall N,M \geq N_0\]
\end{proof}

\section{Topology of $\real^d$}
\subsection{Norms}
\begin{definition}
    A norm on $\real^d$ is a function $||\cdot||: \real^d \rightarrow [0,\infty)$ satisfying the following properties:
    \begin{enumerate}[label=(\roman*)]
        \item $||a|| = 0 \iff a = (0,\dots,0)$
        \item $||ca|| = |c|\cdot||a|| \ \forall c \in \real, a \in \real^d$
        \item $||a+b|| \leq ||a|| + ||b|| \ \forall a,b \in \real^d$
    \end{enumerate}
\end{definition}
The euclidean norm of $\real^d$ is given by
\[||a,\ldots,a_d||_2 = \sqrt{\sum_{i=1}^d a_i^2}\]
The dot prdouct on $\real^d$ is given by
\[(a_1,\ldots,a_n) \cdot (b_1,\ldots,b_n) = \sum_{i=1}^d a_ib_i\]
We also have the $l_1$-norm
\[||(a_1,\ldots, a_n)||_1 =\sum_{i=1}^d |a_i|\]
And the $l_\infty$-norm
\[||(a_1,\ldots, a_d)||_\infty = \max\{|a_1|,\ldots,|a_d|\}\]
\begin{prop}
    Let $a,b\in\real^d$, write $||\cdot||$ for the euclidean norm on $\real^d$. Then
    \begin{enumerate}[label=(\roman*)]
        \item \textbf{Cauchy Schwarz Inequality:} 
        \[|a\cdot b| \leq ||a||\cdot||b||\]
        \item \textbf{Triangle Inequality:}
        \[||a+b|| \leq ||a|| + ||b||\]
        \item $||\cdot||$ is a norm on $\real^d$
    \end{enumerate}
\end{prop}
\begin{proof}
    ~\newline
    \begin{enumerate}[label=(\roman*)]
        \item Consider the equadratic function
        \begin{align*}
            P(t) &= ||a+tb||^2\\
                &=(a+tb)\cdot(a+tb)\\
                &=a\cdot a + 2a\cdot tb + tb\cdot tb\\
                &=||a||^2 + 2t(a\cdot b) + t^2||b||^2
        \end{align*}
        The discriminant of $P(t)$ is less than or equal to 0,
        \begin{align*}
            (2a\cdot b)^2 - 4||a||^2||b||^2 &\leq 0\\
            (a\cdot b)^2 &\leq ||a||^2||b||^2\\
            |a\cdot b| &\leq ||a||\cdot||b||
        \end{align*}
        \item 
        \begin{align*}
            ||a+b||^2 &= ||a||^2 + 2a\cdot b + ||b||^2\\
                    &\leq ||a||^2 + 2|a\cdot b|| + ||b||^2\\
                    &\leq ||a||^2 + 2||a||\cdot||b|| + ||b||^2\\
                    &= (||a|| + ||b||)^2
        \end{align*}
        \[||a+b||^2 \leq (||a|| + ||b||)^2 \implies ||a+b|| \leq ||a||+||b||\]
        \item \textbf{Exercise.} We want to prove $||a|| = 0 \iff a = 0$ and $||ca|| = |c|\cdot||a||$.
    \end{enumerate}
\end{proof}
%----------------------------Lecture 9 ----------------------------------
\chapter{$\real^d$}
\textbf{Recall:} $||(x_1, \dots, x_d)|| \coloneqq \sqrt{x_1^2 + \cdots + x_d^2}$. This is a norm. i.e. 
$$||a + b|| \leq ||a||_2 + ||b||_2 \ \forall a,b \in \real^d$$ 
$$||ca|| = |c|\cdot ||a||_2, \ c \in \real, \ a \in \real^d$$ 
$$||a||_2 > 0, \ \forall a \in \real^d \setminus \{(0,\dots,0)\}$$
$$||(0,\dots, 0)||_2 = 0$$
Other examples of norms: 
\begin{itemize}
    \item $||(x_1, \dots, x_d)||_1 \coloneqq |x_1| + \cdots + |x_d|$
    \item  $||(x_1, \dots, x_d)||_\infty \coloneqq max\{|x_1|, \dots, |x_d|\}$
\end{itemize}
\textbf{Exercise:} For $a \in \real^d$, $$||a||_\infty \leq ||a||_2 \leq ||a||_1 \leq d||a||_\infty (\leq d||a||_2)$$
\textbf{Interesting Fact:} There are other norms. but they are all equivalent in the sense that if $||\cdot||, ||\cdot||'$ are norms on $\real^d$, then $\exists v, R > 0$ such that
$$r||a|| \leq ||a||' \leq R||a||$$
\section{Convergence}
\begin{definition}
Let $(a_n)_{n=1}^\infty$ be a sequence in $\real^d$ and let $L \in \real^d$, we say $(a_n)_{n=1}^\infty$ \textbf{converges} to $L$, and write $\lim_{n \rightarrow \infty} = L$ or $a_n \rightarrow \infty$, if 
$$\lim_{n \rightarrow \infty} ||a_n - L||_2 = 0$$
\end{definition}
\textit{Note: We could define convergence instead using some other norm, say $||\cdot||_1$.}\\[2ex]
If $||a_n - L||_2 \rightarrow 0$, then $||a_n - L||_1 \leq d||a_n - L||_2 \rightarrow 0$
If $||a_n - L||_1 \rightarrow 0$, then $||a_n - L||_2 \leq d||a_n - L||_1 \rightarrow 0$\\[2ex]
in general, if $||\cdot||$ is any norm, then since $||\cdot||$ and $||\cdot||_2$ are equivalent.
$$||a_n - L||_2 \rightarrow 0 \iff ||a_n - L|| \rightarrow 0$$
\textbf{Example:} Say $a_n = \left(\frac{1}{n}, \frac{1}{n}, \dots, \frac{1}{n}\right)$, then 
$$||a_n - L||_2 = \sqrt{1/n^2, + \cdots + 1/n^2} = \sqrt{\frac{d}{n^2}} = \frac{\sqrt{d}}{n}\rightarrow 0$$
$$\therefore a_n \rightarrow L$$
Given a sequence $(a_n)_{n=1}^\infty$ in $\real^d$, we write $a_1 = (a_1^{(1)},a_1^{(2)}, \dots, a_1^{(d)})$ where $a_1^{(1)},a_1^{(2)}, \dots, a_1^{(d)} \in \real$. Similarily, 
$$a_1 = (a_1^{(1)},a_1^{(2)}, \dots, a_1^{(d)})$$
$$a_2 = (a_2^{(1)},a_2^{(2)}, \dots, a_2^{(d)})$$
$$a_3 = (a_3^{(1)},a_3^{(2)}, \dots, a_3^{(d)})$$
$$\vdots$$
$$L = (L^{(1)},L^{(2)}, \dots, L^{(d)}) \in \real^d$$
We get $d$ sequences in $\real$, and $d$ possible limit points $L^{(1)}, \dots, L^{(d)} \in \real$
\begin{prop}
Given $(a_n)_{n=1}^\infty$ and $L$ as above, $a_n \rightarrow L$ as $d \rightarrow \infty$ $\iff$ $a_n^{(i)} \rightarrow L^{(i)}$ in $\real$ as $n \rightarrow \infty$, for $i = 1, \dots, d$.
\end{prop}
\begin{proof}
$\implies$: Suppose $a_n \rightarrow L$, i.e.
\begin{align*}
    ||a_n - L||_2 \rightarrow 0\\
    \intertext{For $x = (x_1, \dots, x_d) \in \real^d$}
    ||x||^2_2 = x_1^2 + \cdots x_d^2 \geq x_i^2 = |x_i|^2
    \therefore |x_i \leq ||x||_2
\end{align*}
Applying this to (*), we get 
$$0 \leq |a_n^i - L^{(i)}| \leq ||a_n - L ||_2 \rightarrow 0$$
So by the squeeze theorem, 
$$|a_n^{(i)} - L^{(i)}| \rightarrow 0$$\\[2ex]
$\implies$: Suppose $a_n^{(i)} \rightarrow L^{(i)}$ for $i = 1, \dots, d$.
$$||a_n - L||^2_2 = (a_n^{(i)} - L^i)^2 + \cdots + (a_n^d - L^(d)) \rightarrow 0$$
By algebra of limits,
$$\therefore ||a_n - L||_2 \rightarrow 0$$
\end{proof}
\textbf{Example:} $a^n = ((-1)^n, \frac{1}{n}) \in \real^2$. Does $(a_n)$ converge? No, since $(-1)^n$ does not converge.

\begin{definition}
    A sequence $(a_n)_{n=1}^\infty$ in $\real^d$ is \textbf{Cauchy} if  $\forall \epsilon > 0$, $\exists n_0 \in \nat_{\geq 1}$ such that 
    $$||a_n - a_m||_2 < \epsilon \ \forall m,n \geq n_0$$
\end{definition}

\begin{theorem}[Cauchy Convergence Criterion for $\real^d$]
Let $(a_n)_{n=1}^\infty$ be a sequence in $\real^d$. It converges $\iff$ it is Cauchy.
\end{theorem}

\begin{proof}
    $\implies$: Suppose $a_n \rightarrow L \in \real^d$. To show it is Cauchy, let $\epsilon > 0$. $||a_n - L||_2 \rightarrow 0$, so $\exists n_0 \in \nat_{\geq 1}$ such that 
    $$||a_n - L||_2 < \frac{\epsilon}{2} \ \forall n \geq n_0)$$
    Then if $m,n \geq n_0$, 
    \begin{align*}
        ||a_m - a_n||_2 &= ||a_m - L + L - a_n||_2\\
        &\leq ||a_m - L||_2 + ||L - a_n||_2\\
        &< \frac{\epsilon}{2} + \frac{\epsilon}{2} = \epsilon
    \end{align*}
        $\therefore (a_n)_{n=1}^\infty$ is Cauchy.
    $\implies$: Suppose $(a_n)_{n=1}^\infty$ is Cauchy, write 
    $$a_n = (a_n^{(1)}, \dots, a_n^{(d)}$$
    For any $m,n \in \nat_{\geq 1}$,
    $$|a_n^{(i)} - a_m^{(i)} \leq ||a_n - a_m||_2$$
    $\therefore (a_n^{(i)})_{n=1}^\infty$ is Cauchy in $\real$. So by the Cauchy Convergence Criterion, $\exists L^{(i)} \in \real$, such that $a_n^{(i)} \rightarrow L^{(i)}$. By the previous proposition,
    $$a_n \rightarrow (L^{(1)}, \dots, L^{(d)})$$
\end{proof}
\begin{definition}
    $S \subseteq \real^d$ is \textbf{bounded} if $\exists M > 0$ such that 
    $$||x|| \leq M \ \ \forall x \in S$$
\end{definition}
A sequence $(a_n)_{n=1}^\infty$ in $\real^d$ is bound if
$\{a_n: n \in \nat_{\geq 1}\}$ is a bounded set.
\begin{theorem}[Bolzano-Weierstrass for $\real^d$]
    If $(a_n)_{n=1}^\infty$ is a bounded sequence in $\real^d$, then it has a subsequence $(a_{n_k})_{n=1}^\infty$ that converges.
\end{theorem}
\begin{proof}
    Write $a_n = (a_n^{(1)}, \dots, a_n^{(d)})$.\\ 
    We will prove it by indunction on $d$. For $d = 1$, this is the Bolzano-Weierstrass theorem for $\real$ For $d>1$ write
    \[b_n \coloneq (a_n^{(1)}, \ldots, a_n^{(d-1)}) \in \real^{d-1}\]
    By the induction hypothesis, $b_n$ has a subsequence $(b_{n_k})_{k=1}^\infty$ that converges. Let $L \in \real^{d-1}$ be the limit of this subsequence. Then $L \in \real^{d-1}$ is the limit of $b_n$. $(a^{(d)}_{n_k})^\infty_{k=1}$ is a bounded sequence in $\real$, so it has a subsequence $(a^{(d)}_{n_{k_j}})^\infty_{j=1}$ that converges. Let $L^{(d)} \in \real$ be the limit of this subsequence. Then $L = (L^{(1)}, \ldots, L^{(d-1)}, L^{(d)})$ is the limit of $a_n$.
\end{proof}
\chapter{Open and Closed Sets in $\real^d$}
Roughly, an open set is one that we draw with dotted lines. The line represents a "boundary" that is the not in the set. This is not a rigorous definition.
\begin{definition}[Open Ball]
    Let $a\in \real^d$, $r > 0$. The \textbf{open ball} of radius $r$ centered at $a$ is $$B(a;r) \coloneqq \{x \in \real^d: ||x-a||_2 < r\}$$
\end{definition}
\textbf{Relation to Convergence:} If $a_n \rightarrow L$, then this means that $||a_n - L ||_2 < \epsilon$ for all $n$ large. So, $a_n \in B(L;\epsilon)$
\begin{definition}[Open Sets]
    A set $U \subseteq \real^d$ is \textbf{open} if $\forall a \in U$, $\exists r > 0$, such that $B(a; r) \subseteq U$
\end{definition}
\textbf{Idea:} If $a \in U$, then $a$ is not on the boundary but it is truly "inside" the set, so we can fit a ball containing a in the set.
\begin{definition}[Closed Sets]
    A set $k \in \real^d$ is \textbf{closed} if its complement $\real^d \setminus k$ is open. 
\end{definition}
\textbf{Example:} $U \subseteq (0,1)$. Is this open? Yes.
\begin{proof}
    Let $a \in U$. We let $r := \min\{|a-0|, |a-1|\}$ (We do this so that $r$ is at most the distance to the closest bound, i.e. if $a$ is closer to $0$, then the radius $r$ cannot be $|a-1|$)then 
    $$B(a ; r) = (a-r,a+r) \subseteq (0,1) = U$$
\end{proof}
\textbf{Example:} $U \coloneqq [0,1]$. Is this open? No.
\begin{proof}
    Let $a \coloneqq 0 \in U$. The for any $r > 0$, $\exists z \in B(a ; r) = (-r,r)$ s.t $z < 0$, so $z \not\in U$. Therefore $B(a;r) \subseteq U$
\end{proof}
Is $U$ closed? This is the same as asking if $\real \setminus U = (-\infty, 0) \cup (1,\infty)$ is open. 
This is open.
\begin{proof}
    Let $a \in (-\infty, 0) \cup (1,\infty)$.
    \begin{itemize}
        \item \textbf{Case 1:} $a \in (-\infty, 0)$. Set $r := |a|$, so 
        $$B(a;r) = (a-r,a+r) = (2a, 0) \subseteq U$$
        \item \textbf{Case 2:} $a \in (1, \infty)$ similar.
    \end{itemize}
\end{proof}
Therefore $U = [0,1]$ is closed.\\[3ex]
\textbf{Example:} Is $U \coloneqq (0,1]$ open? No, for any $r > 0$
$$B(1;r) \not\subseteq U$$
Therefore, it is not open.\\[2ex]
\begin{center}
    \textit{\textbf{Note:} Sets are not always open or closed. Most sets are neither open nor closed.}
\end{center}
This set $U$ is one such example $U$ is not closed since $\real \setminus U = (-\infty, 0] \cup (1 \infty)$
$0 \in \real \ U$ but $\forall r >0$, $B(0;r) \not\subseteq R \setminus U$\\[3ex]
\textbf{Example:} For any $a \in \real^d$, $r > 0$ $B(a;r)$ is an open set.
\begin{proof}
    Let $x \in B(a;r)$, so $||x-a||_2 < r$. Set 
    $$r_0 \coloneqq r - ||x-a||_2 > 0$$
    \textbf{Claim:} $B(x; r_0) \subseteq B(a;r)$
    To see this, let $y \in B(x;r_0)$ so $|y-x||_2 < r_0$. So,
    \begin{align*}
        ||y-a||_2 &\leq ||y-x||_2 + ||x - a||_2\tag{$\triangle$-inequality} \\
        &< r_0 + ||x - a||_2\\
        &= r
    \end{align*}
\end{proof}
\begin{prop}
    \begin{enumerate}[label=(\roman*)]
        \item $\emptyset$, $\real^d$ are both open in $\real^d$
        \item If $U_1, U_2, \dots, U_n \subseteq \real^d$ are all open, then so is $U_1 \cap U_2 \cap \cdots \cap U_n$.
        \item If $U_a \subseteq \real^d$ is an open set for all $\alpha \in I$, (I is some index set) then 
        $$\bigcup_{a\in I} U_a$$
        is open.
    \end{enumerate}
\end{prop}
\begin{proof}
    \textit{(i),(ii) are exercises}.\\[2ex]
    \textbf{(iii):} Set 
    $$V \coloneqq \bigcup_{\alpha \in I} U_a$$
    Let $a \in V$. This means $\exists \alpha \in I$ such that $a \in U_\alpha$. $U_\alpha$ is open so $\exists r > 0$ s.t $B(a;r) \subseteq U_\alpha$. $U_\alpha \leq \bigcup_{\alpha \in I} U_\alpha = V$ So $B(a;r) \subseteq V$ as required.
\end{proof}
\noindent
\textbf{Example:} For any $n \in \nat_{\geq 1}$.
$$\left(\frac{-1}{n},\frac{1}{n}\right) = B(0;\frac{1}{n})$$ is open in $\real$. The intersection of these open sets is 
$$\bigcap^\infty_{n = 1} \left(\frac{1}{n},\frac{-1}{n}\right) = \{0\} $$ which is not open. This shows that openess is not preserved by infinite intersections. \\[2ex]
\textbf{Example:} Let $$U \coloneqq \{(x,y) \in \real^2 : x > 0 , y >0\}$$
$U$ is open but not closed.\\[1ex]
$$U = V \cap W$$
where 
\begin{align*}
    V \coloneqq \{(x,y) : x > 0\} &&
    W \coloneqq \{(x,y): y >0\}
\end{align*}
To show $V$ is open, let $a = (x,y) \in V$. Set $r \coloneqq x > 0$. Then if $(w,z) \in B(a;r)$. Then 
$$|w-z| \leq ||(w,z)-a||_2 < r = x$$
$$\therefore w > x -x = 0$$
So $(w,z) \in U$. Similarly, $W$ is open. Therefore $U$ is open.\\[2ex]
\textbf{Not Closed:} Exercise.
\begin{prop}
    Let $K \subseteq \real^d$. $K$ is closed $\iff$ for any subsequence $\infseq{a}$ in $K$, If it converges, then
    $$lim_{n \rightarrow \infty} a_n \in K$$
\end{prop}
\begin{proof}
    ($\implies$) Suppose $K$ is closed. Let $\infseq{a}$ be a sequence in $K$ s.t 
    $$L \coloneq lim_{n \rightarrow \infty} a_n$$
    exists. Suppose for a contradction $L \not\in K$. This means $L \in \real^d \setminus K$, which is open. So $\exists r > 0$ such that 
    $$B(L;r) \subseteq \real^d \setminus K$$
    Since $a_n \rightarrow L$, we must have $a_n \in B(L ; a)$ for some $n$ (in fact, for all n sufficiently large. So $a_n \in B(L;r) \subseteq \real^d \setminus K$. Therefore $a_n \not\in K$, which is a contradiction. \\[1ex]
    ($\impliedby$) Suppose $K$ is not closed, and we'll prove $\exists \infseq{a}$ in $K$ such that $a_n \rightarrow L \not\in K$. Since $K$ is not closed, $\real^d \setminus K$ is not open. So $\exists L \in \real^d \setminus K$ such that $\forall r > 0$
    $$B(L;r) \not\subseteq \real^d \setminus K$$
    For each $n \in \nat_{\geq1}$, we can fine $a_n \in B(L; \frac{1}{n})$ such that $a_n \not\in \real^d \setminus K$. So $a_n \in K$. This gives a sequence $\infseq{a}$ in $K$ and 
    $$||a_n - L||_2 < \frac{1}{n} \rightarrow 0$$
    Therefore by the Squeeze Theorem,
    $$||a_n - L||_2 \rightarrow 0 \implies a_n \rightarrow L$$
    $L \in \real^d \setminus K$, so $L \not\in K$.
\end{proof}
\begin{definition}
    Let $A \subseteq \real^d$ and let $a \in \real^d$, a is: 
    \begin{enumerate}[label=(\roman*)]
        \item an \textbf{interior point} if $\exists r > 0$ s.t $B(a;r) \subseteq A$
        \item an \textbf{accumulation point} if $\exists$ a sequence $\infseq{a}$ in $A$ s.t $a_n \rightarrow a$
        \item a \textbf{boundary point} if it is an accumulation point and it is not an interior point.
        $$A^\circ \coloneqq \{All \ interior \ points\}$$
        $$\bar{A} \coloneqq \{All \ accumulation \ points\}$$
        $$\partial A \coloneqq \{All \  boundary 
 \ points\} = \bar{A} \setminus A^\circ$$
    \end{enumerate}
    \begin{center}
        \textit{\textbf{Note:} The set of interior points, accumulation points, and boundary points are referred to as the \textbf{interior} of A, the \textbf{closure} of A, and the \textbf{boundary} of A respectively}
    \end{center}
\end{definition}
\textbf{Example:} $A \coloneqq (0, 1] \cup \{2\}$\\[1ex]
$$A^\circ = (0,1)$$
$$\bar{A} = [0,1] \cup \{2\}$$
$$\partial A = \{0,1,2\}$$
\textbf{Example:} $A \coloneqq \mathbb{Q}$\\[1ex]
Since any open interval contains irrational numbers, we have
$$A^\circ = \empty set$$
Proposition from chapter 2,
$$\bar{A} = \real$$
$$\partial A = \real$$

\chapter{Compactness}
\begin{definition}
    A set $A \subseteq \real^d$ is (sequentially) compact if every sequence $\infseq{a}$ in $A$ has a subsequence $(a_{n_k})^\infty_{k = 1}$ that converges to a point in $A$.
\end{definition}
\textbf{Example 1:} Is [0,1] compact? Yes.
\begin{proof}
    \textbf{Recall:} Bolzano-Weierstrass theorem states bounded sequence has a convergent subsequence. \\[1ex]
    Therefore, every sequence $\infseq{a}$ in $[0,1]$ has a subsequence $\seq{a_n}{k}{\infty}$ that converges. So
    $$0 \leq a_{n_k} \leq 1 \implies 0 \leq \lim_{k \rightarrow \infty} a_{n_k} \leq 1$$
    $$\therefore L \in [0,1]$$
\end{proof}
\textbf{Example 2:} Is (0,1) compact? No.
\begin{proof}
    By counter example, let $a_n \coloneqq \frac{1}{n+1}$, so $a_n \rightarrow 0$. Therefore
    for all subsequences of $a_n$, $a_{n_k} \rightarrow 0$. So there exists no subsequence which converges
    to a point in (0,1).
\end{proof}
\textbf{Example 3:} Is $[0,\infty)$ compact? No.
\begin{proof}
    The Bolzano-Weierstrass theorem does not apply since $[0,\infty)$ is unbounded.
    Set $a_n \coloneqq n$, then $a_n \rightarrow \infty$, so it has no bounded subsequence
    and therefore no convergent subsequences.
\end{proof}
\begin{theorem}[Heine-Borel]
    Let $A \subseteq \real^d$. $A$ is compact $\iff$ A i$s$ closed and not bounded.
\end{theorem}
\begin{proof}
    ($\implies$) Similar to example 1. Assume $A$ is closed and bounded. Let $\infseq{a}$ be a sequence in $A$.
    The sequence is bounded since $A$ is, so by the Bolzano-Weierstrass theorem for $\real^d$, it has a subsequence $\seq{a_n}{k}{\infty}$ that converges to some $L \in \real^d$. $a_{n_k} \in A$ $\forall k$ and $a_{n_k} \rightarrow L$ and $A$ is closed, so by the sequential characterization of closedness, $L \in A$, therefore $A$ is compact.\\[2ex]
    ($\impliedby$) Assume $A$ is compact. To show $A$ is closed, assume for a contradction that $A$ is not closed. Therefore there exists a sequence $\seq{a}{n}{\infty}$ in $A$ such that $a_n \rightarrow L\not\in A$. Then for any subsequence $\seq{a_n}{k}{\infty}$, we have 
    $$a_n \rightarrow L \not\in A$$
    This contradicts that $A$ is compact, therefore $A$ is closed. \\[1ex]
    Tow show $A$ is bounded, assume for a contradiction that $A$ is not bounded. Then $\forall n \in \nat_{\geq1}$, there exists $a_n \in A$ such that $||a_n||_2 \geq n$. This gives a sequence $\seq{a}{n}{\infty}$. Since $A$ is compact, it has a subsequence $\seq{a_n}{k}{\infty}$ that converges. But 
    \[||a_{n_k}||_2 \geq n_k \rightarrow \infty\]
    So $\seq{a_n}{k}{\infty}$ is unbounded, which is a contradiction.
\end{proof}
\begin{prop}
~\newline
    \begin{enumerate}[label=(\roman*)]
        \item If $k_1, \ldots, k_n \subseteq \real^d$ are compact, then $\bigcup_{i=1}^n k_i$ is compact.
        \item If $k_1, \ldots, k_n \subseteq \real^d$ are compact, then $\bigcap_{i=1}^n k_i$ is compact.
    \end{enumerate}
\end{prop}
\begin{proof}
    \textbf{Exercise:}
    \begin{enumerate}[label=(\roman*)]
        \item Assume $A \coloneqq \bigcup_{i=1}^n k_i$. Let $\infseq{a}$ be a sequence in $A$. Then there exists $i \in \{1, \ldots, n\}$ such that $a_n \in k_i$. Since $k_i$ is compact, it has a subsequence $\seq{a_n}{k}{\infty}$ that converges to some $L \in \real^d$. $L \in k_i$ and $k_i \subseteq A$, so $L \in A$. Therefore $A$ is compact.\\[2ex]
        \item Assume $A \coloneqq \bigcap_{i=1}^n k_i$. Let $\infseq{a}$ be a sequence in $A$. Then $a_n \in k_i$ $\forall i \in \{1, \ldots, n\}$. Since $k_i$ is compact, it has a subsequence $\seq{a_n}{k}{\infty}$ that converges to some $L \in \real^d$. $L \in k_i$ $\forall i \in \{1, \ldots, n\}$ and $k_i \subseteq A$, so $L \in A$. Therefore $A$ is compact.
    \end{enumerate}
\end{proof}
\begin{definition}
    $A \subseteq \real^d$ is \textbf{compact} if for any collection
    \[U_\alpha: \alpha \in I\]
    of open sets such that 
    \[A \subseteq \bigcup_{\alpha \in I} U_\alpha\]
    There exists finitely many indeces $\alpha_1, \ldots, \alpha_n$ such that 
    \[A \subseteq \bigcup_{i=1}^n U_{\alpha_i}\]
\end{definition}
\chapter{Limits of a Function of Continuous Variables}
A sequence is a function $\nat \rightarrow \real$. Here, we'll consider function 
that are going from $\real \rightarrow \real$ (or $\real^d \rightarrow \real^m$).
\begin{definition}
    Let $X \subseteq \real^d$, $a \in \real^d$ a limit point of $X$.
    $f: X \rightarrow \real^m$, $L \in \real^m$. We say the limit of $f$
    as $X$ approaches $a$ is $L$ if 
    $$\forall \epsilon > 0, \exists \delta > 0 \ s.t \ \forall x \in X$$
    $$x\in B(a;\delta) \wedge x \neq a \implies ||f(x)-L||_2 < \epsilon$$
\end{definition}

The idea is like the definition of convergence of a sequence, except we replace $n \geq n_0$
(which captures "$n$ is sufficiently large") with $x \in B(a;\delta)$, $x \neq a$ (which captures
"$x$ is close to, but not equal to $a$). In other words, the definition says
that if $x$ is close to (but not equal to) $a$
 then $f(x)$ is close to $L$.\\[2ex]

 \textbf{Why "not equal to"?:} Often we consider the limit as $x$ approaches
 $a$ when $f(a)$ is not defined. Other times we compare the limite to $f(a)$.
 So we do not want to use $f(a)$ in the definition of the limit. \\[2ex]

 \textbf{Notation:} We write 
 $$\lim_{x \rightarrow a} f(x) = L$$
 or 
 $$f(x) \rightarrow L \ as \ x \rightarrow d$$
 to mean that the limit of $f$ is $L$ as $x$ approaches $a$.\\[2ex]
 \textbf{Example:} $f: \real \rightarrow \real$. $f(x) \coloneqq 3x - 2$. 
 Let $a \in \real$. Claim
 $$\lim_{x\rightarrow a} f(x) = 3a - 2$$
 \begin{proof}
    Let $\epsilon > 0$. Consider 
    \begin{align*}
        |f(x) - (3a-2)| &= |3x-2 - 3a + 2|\\
        &= 3|x-a|
    \end{align*}
    We want this $< \epsilon$, set $\delta \coloneqq \frac{\epsilon}{3}$. Then if
    $x \in B(a; \delta) = (a - \delta, a + \delta)$ (i.e $|x-a| < \delta$)
    then 
    $$|f(x) - (3a - 2)| = 3|x-a| < 3 \delta = \frac{3\epsilon}{3} = \epsilon$$
 \end{proof}
\textbf{Example:} $g: \real \rightarrow \real$. $g(x) \coloneqq x^2$. Claim: 
$$\lim_{x\rightarrow a} g(x) = a^2$$
\begin{proof}
    Let $\epsilon > 0$ be given.
    \begin{align*}
        |g(x) - a^2| &= |x^2 - a^2| \\
        & = |x - a| |x + a|
    \end{align*}
    What happens if $x$ is close to $a$? Intuitively, $|x + a|$ is close to $|a+a|$
    and $|x-a|$ is small.
    \begin{align*}
        |x+a| &= |x -a + a + a| \leq |x - a| + |a + a|\\
              &< 2|a| + \delta \tag{if $|x-a|<\delta$}\\
              &\leq 2|a| + 1\tag{$\delta \leq 1$}
    \end{align*}
    Then, 
    \begin{align*}
        |x^2 - a^2| &= |x-a||x+a|\leq |x-a|(2|a| + 1)\\
                    &< \delta(2|a| + 1)\tag{if $|x-a|<\delta$}\\
                    &\leq \epsilon\tag{if $\delta \leq \frac{\epsilon}{2|a| + 1}$}
    \end{align*}
    \begin{center}
        \red{\textbf{Important:} Do not define $\delta$ in terms of $x$ or $\delta$!
        We can use $a$ here since $a$ is constant}
    \end{center}
    So we set $\delta \coloneqq \min \{1, \frac{\epsilon}{2|a| + 1}\}$
    Then $\delta \leq \frac{\epsilon}{2|a| + 1}$ and $\delta \leq 1$. So
    if $|x-a| < \delta$. Then from the work above, $|x^2-a^2| < \epsilon$ as required.
\end{proof}

\textbf{Note:} In proofs where we have $\delta - \epsilon$, we often use
$$\delta \coloneqq \min \{\ldots\}$$
In proofs where we have $n_0 - \epsilon$, we often use
$$n_0 \coloneqq \max \{\ldots\}$$
\begin{prop}[Uniqueness of Limits]
    Let $f: X \rightarrow \real^m$ ($X \subseteq \real^d$), $a \in \real^d$
    a limit point of $X$, $L, L' \in \real^m$. If the limit of $f$ as $x \rightarrow a$ is L
    and the limit of $f$ as $x \rightarrow a$ is $L'$, then $L = L'$
\end{prop}
\begin{proof}
    By contradction. Suppose $L \neq L'$. So 
    $$||L - L'||_2 > 0$$
    Set 
    $$\epsilon \coloneqq \frac{||L - L||_2}{2} > 0$$
    Since $f(x) \rightarrow L$ as $x \rightarrow a$, $\exists \delta > 0$
    such that if $x \in X \cap B(a; \delta)\setminus \{a\}$, then
    $$||f(x) - L||_2 < \epsilon$$
    Since $f(x) \rightarrow L'$ as $x \rightarrow a$, $\exists \delta' > 0$ such that
    $$x \in X \cap B(a;\delta')\setminus \{a\} \implies ||f(x) - L'||_2 < \epsilon$$
    Let $\delta_0 \neq \min \{\delta, \delta'\}$. Let 
    $$x \in X \cap B(a;\delta_0) \setminus \{a\}$$
    Then
    $$x \in X \cap B(a; \delta) \setminus \{a\}$$
    So,
    \begin{align*}
        ||f(x) - L||_2 &< \epsilon\\
        ||f(x) - L'||_2 &< \epsilon
    \end{align*}
    So, 
    \begin{align*}
        ||L - L'||_2 \leq ||L - f(x)||_2 + ||f(x) - L'|| &< \epsilon + \epsilon\\
        &= ||L - L'||_2
    \end{align*}
    And thus, a contradction. 
\end{proof}
\begin{prop}[Sequential Characterization of Limits]
    Let $X \subseteq \real^d$, $a \in \real^d$, a limit point of $X$.
    $f: X \rightarrow \real^m$, $L \in \real^m$.\\[1ex]
    $\lim_{x\rightarrow a} f(x) = L \iff$ for every sequence $\infseq{x}$ in $X$
    such that $x_n \rightarrow a$, we have 
    $$\lim_{n \rightarrow \infty} f(x_n) = L$$
    
\end{prop}
\begin{proof}
    ($\implies$) Suppose $\lim_{x\rightarrow a} f(x) = L$. Let $\infseq{x}$ be a sequence in
    $X \setminus \{a\}$ such that $x_n \rightarrow a$. We must show that $f(x_n) \rightarrow L$.\\[1ex]
    Let $\epsilon > 0$ be given. Since $f(x) \rightarrow L$ as $x \rightarrow a$, $\exists \delta$
    such that 
    $$x \in X \cap B(a; \delta) \setminus \{a\} \implies ||f(x) - L||_2 < \epsilon$$
    Since $x_n \rightarrow$, using $\delta$ in place of $\epsilon$, $\exists n_0$ such that
    $\forall n \geq n_0$, $||x_n a||_2 < \delta$. i.e. $x_n \in B(a,\delta)$. Also 
    $x_n \in X \setminus \{a\}$ Therefore, 
    $$||f(x) - L ||_2 < \epsilon$$
    ($\impliedby$) Suppose $\forall$ sequences $\infseq{x}$ ins $X \setminus \{a\}$ converging to $a$, $f(x) \rightarrow L$,
    and for a contradction, suppose $$f(x) \not\rightarrow L$$
    We negate "$f(x) \rightarrow L$" to get that $\exists \epsilon > 0$ such that $\forall \gamma > 0$,
    $\exists x \in X \cap B(a;\delta) \setminus \{a\}$ such that $||f(x) - L||_2 \geq \epsilon$.\\[1ex]
    This gives a sequence $\infseq{x_n}$ in $X \setminus \{a\}$, $||x_n - a||_2 \leq \frac{1}{n}$ $\forall n$,
    so by the squeeze theorem 
    $$||x_n - a||_2 \rightarrow 0$$
    Since $||f(x_n) - L||_2 \geq \epsilon$, $f(x_n) \not\rightarrow L$. This is a contradction.
\end{proof}
\begin{center}
    \textbf{Note:} if $\lim_{n\rightarrow \infty} f(x_n) = L$ for \emph{some} sequence
    $\infseq{x}$ in $X \setminus \{a\}$ convering to a, it \emph{does not} follow
    that $\lim_{x \rightarrow a} f(x) = L$
\end{center}
\textbf{Example:}
$$f(x) \coloneqq\begin{cases}
     0 \text{   if $x = \frac{1}{n}$, $n \in \nat_{\geq 1}$}\\
     1 \text{   otherwise}
\end{cases}$$
$\lim_{x\rightarrow 0}f(x)$ does not exist but $\lim_{n\rightarrow \infty}f(\frac{1}{n}) = 0$
\begin{prop}[Algebra of Limits]
    Let $x \subseteq \real^d$, $a \in \real^d$ a limit point of $X$, $f: X \real^m$
    ,$g: X \rightarrow \real^m$, $L,K \in \real^m$. Suppose $\lim_{x\rightarrow a} f(x) = L$, 
    $lim_{x\rightarrow a} g(x) = K$
    \begin{enumerate}[label=(\roman*)]
        \item $$\lim_{x \rightarrow 0} f(x) + g(x) = L + K$$
        \item $$\lim_{cf(x)} = cL$$
        \item If $m = 1$, $$lim_{x\rightarrow a}f(x)g(x) = LK$$
        \item If $m=1$, $g(x) \neq 0$ $\forall x \in X$, $K \neq 0$. Then
        $$lim_{x\rightarrow a}\frac{f(x)}{g(x)} = \frac{L}{K}$$
    \end{enumerate}
\end{prop}
\begin{proof}
    \begin{enumerate}[label=(\roman*)]
        \item Use Sequential Characterization: Let $\infseq{x}$ be in $X \setminus \{a\}$
        such that $x \rightarrow a$. Then $f(x_n) \rightarrow L$ and $g(x_n) \rightarrow K$
        So by algebra of limits for sequences, 
        $$f(x_n) + g(x_n) = L + K$$
        $$\therefore f(x) + g(x) \rightarrow L + K$$
        \item \textbf{Exercise.}
        \item \textbf{Exercise.}
        \item \textbf{Exercise.}
    \end{enumerate}
\end{proof}
\begin{theorem}[Squeeze Theorem]
    Let $X \subseteq \real^d$, $a \in \real^d$ a limit point of $X$, 
    $f,g,h: X \rightarrow \real$
    $$f(x) \leq g(x) \leq h(x) \ \forall x \in X$$
    and 
    $$\lim_{x\rightarrow a} f(x) =  \lim_{x \rightarrow a} h(x) = L$$
    Then
    $$\lim_{x\rightarrow a} g(x) = L$$
\end{theorem}
\begin{proof}
    \textbf{Exercise}
\end{proof}
If $f: X \rightarrow \real_m$, We can define functions 
$$f_1, \ldots, f_m: X \rightarrow \real$$
by 
$$(f_1(x), \ldots, f_m(x)) = f(x)$$
$f_1, \ldots, f_m$ are called the \emph{component functions} of $f$.
\begin{prop}
    Let $X \in \real^d$, $a \in \real^d$ a limit point of $X$, $f: X \rightarrow \real^m$,
    $f_1,\ldots, f_m$ its component functions. $L = (L_1, \ldots, L_m) \in \real^m$. Then
    $$\lim_{x\rightarrow a} f(x) = L \iff \lim_{x\rightarrow a} f_i(x) = L_i \ \forall 1 \leq i \leq m$$
\end{prop}
\begin{proof}
    \textbf{Exercise.}
\end{proof}
\begin{definition}
    Let $X \subseteq \real$, $a \in \real$, $f: X \rightarrow \real^d$.
    \begin{itemize}
        \item If $a$ is a limit point of $X \cap (a, \infty)$ then we write
            $\lim_{x\rightarrow a^+} f(x) = L$ to mean that 
            $$\lim_{x \rightarrow a} g(x) = L$$
            where 
            $$g = f{\big|}_{X \cap (a, \infty)}$$
        \item If $a$ is a limit point of $X \cap (-\infty,a)$ then we write
        $\lim_{x\rightarrow a^+} f(x) = L$ to mean that 
        $$\lim_{x \rightarrow a} g(x) = L$$
        where 
        $$g = f \mid_{X \cap (-\infty, a)}$$
    \end{itemize}
\end{definition}
\textbf{Example:}
$$f(x)\coloneqq \begin{cases}
    -1, x < 0\\
    0, x = 0\\
    1, x > 0
\end{cases}$$
$$\lim_{x \rightarrow 0^+} f(x) = 1 \neq \lim_{x \rightarrow 0^-} f(x) = -1$$

%---------------------- Lecture 13 ----------------------%
\chapter{Continuity}
\section{One-Sided Limits}
\textbf{Recall:} For $X \subseteq \real$, $f: X \rightarrow \real^m$, $a \in \real$
\begin{itemize}
    \item If $a$ is a limit point of $X \cap [a, \infty]$, then 
    $$\lim_{x \rightarrow a^+} f(x) \coloneqq \lim_{x\rightarrow a}f_{X\cap [a,\infty)}(x)$$
    $\lim\limits_{x\rightarrow a^-}f(x)$ is defined similarly. 
\end{itemize}
In other words, $\lim\limits_{x\rightarrow a^+}f(x) = L$ means $\forall \epsilon>0$, $\exists \delta>0$ such that if $a < x < a + \delta$ and $x \in X$, then 
\[||f(x) - L||_2 < \epsilon\]
\section{Continuity}
\begin{definition}
    Let $f: X\rightarrow \real^m$ where $X \subseteq \real^d$. Let $a \in X$.
    \begin{itemize}
        \item If a is not an isolated point, then we say f is continuous at a if 
        \[\lim_{x\rightarrow a}f(x) = f(a)\]
        \item If a is an isolated point, then we always say $f$ is \emph{continuous} at a.
    \end{itemize}
\end{definition}
\begin{definition}[$\delta-\epsilon$ Characterization of Continuity]
    f is continuous at a $\iff$ $\forall \epsilon > 0$, $\exists \gamma > 0$ s.t if
    \[x \in X \cap B(a; \delta)\]
    then 
    \[||f(x) - f(a)||_2 < \epsilon\]
    In other words if x is close to a, the f(x) is close to f(a). This holds regardless of whether or not a is isolated.
\end{definition}
\noindent
\textbf{Example:} $f: \real \mapsto \real$ $f(x) \coloneqq x^2$. For any $a \in \real$, we proved last lecture that 
\[\lim_{x\rightarrow a}f(x) = a^2 = f(a)\]
Therefore, $f$ is continuous at $a$.\\[2ex]
\textbf{Example:} 
\[f(x) \coloneqq \begin{cases}
    1 \text{  if } x \leq 1\\
    2 \text{  if } x > 1
\end{cases}\]
For this function, 
\[\lim_{x \rightarrow 1^+ f(x) = 2 \neq f(x) = 1}\]
Therefore, $f$ is not continuous at 1.\\[2ex]
\textbf{Example:} 
\[f(x) \coloneqq \begin{cases}
    \sin\left(\frac{1}{x}\right) \text{  if } x > 0\\
    L \text{  if } x = 0 \text{ for some $L \in \real$}\\
\end{cases}\]
No matter how we choose $L$, $f$ is not continuous at 0. 
\begin{proof}
    We want to show $\lim\limits_{x \rightarrow 0}f(x)$ does not exist. Consider the sequence 
    \[\left(\frac{1}{2\pi n}\right)^\infty_{n=1} \rightarrow 0\]
    and $f\left(\frac{1}{2\pi n} = 0 \rightarrow 0\right)$. But we also have 
    \[\left(\frac{1}{2\pi n + \pi/2}\right)_{n=1}^\infty \rightarrow 0\]
    and $f\left(\frac{1}{2\pi n + \pi/2}\right) = 1 \rightarrow 1$. So if $\lim\limits_{x \rightarrow 0}f(x)$ exists, then $f(a_n) \rightarrow L$ for a sequence $\infseq{a_n}$ in $(0, \infty)$ such that $a_n \rightarrow 0$. So $L = 0$ and $L = 1$, therefore a contradiction
\end{proof}
\textbf{Example:} 
\[f(x) \coloneqq \begin{cases}
    1 \text{  if } x \in \mathbb{Q}\\
    0 \text{  if } x \notin \mathbb{Q}
\end{cases}\]
$f$ is not continuous at any point.
\begin{proof}
    Let $a \in \real$. We'll prove $\lim\limits_{x \rightarrow a}f(x)$ does not exist.
    \[\exists \infseq(x) \in \mathbb{Q} \text{ s.t } x_n \rightarrow a\]
    so $f(x_n) = 1 \rightarrow 1$
    \[\exists \infseq(y) \in \real \setminus \mathbb{Q} \text{ s.t } y_n \rightarrow a\]
    Since we can find $z_n \in \mathbb{Q}$ such that $z_N \rightarrow a + \sqrt{2}$, so $y \coloneqq z_n - \sqrt{2} \in \real \setminus \mathbb{Q}$, $y_n \rightarrow a$. So $f(y_n) = 0 \rightarrow 0$.  Therefore, the $\lim\limits_{x \rightarrow a}f(x)$ does not exist.
\end{proof}
\textbf{Example:}
\[f(x) \coloneqq \begin{cases}
    x\sin\left(\frac{1}{x}\right) \text{ if } x \neq 0\\
    0 \text{ if } x = 0
\end{cases}\]
$f$ is continuous. 
\begin{proof}
    Use squeeze theorem, if $x \geq 0$, then $-x \leq f(x) \leq x$. So 
    \[-|x| \leq f(x) \leq |x|\]
    \[\lim_{x\rightarrow 0}|x| = 0 = \lim_{x\rightarrow 0}-|x|\]
    \[\therefore \lim_{x\rightarrow 0} f(x) = 0 = f(0)\]
\end{proof}
\begin{prop}
    Let $X \subseteq \real^d$, $Y \subseteq \real^m$, and $f: X \mapsto Y$, $g: Y \rightarrow \real^n$, $a \in X$. If $f$ is continuous at a and g is continuous at f(a), then $g \circ f$ is continuous at a.
\end{prop}
\begin{proof}
    Use the $\delta-\epsilon$ characterization. Let $\epsilon > 0$ be given. Since $g$ is at $f(a)$, $\exists \eta > 0$ (using $\eta$ instead of $\delta$), such that if $y \in B(f(a); \eta) \cap Y$, then 
    \[||g(x) - g(f(a))||_2 < \epsilon\]
    Since $f$ is continuous at $a$, $\exists \delta > 0$ such that if $x \in B(a; \delta) \cap X$, then
    \[||f(x) - f(a)||_2 < \eta\]
    In other words, $f(x) \in B(f(a); \eta)$. Therefore if $x \in B(a; \delta) \cap X$, then since  $f(x) \in B(f(a); \eta)$, 
    \[||g(x) - g(f(a))||_2 < \epsilon\]
    as required.
\end{proof}

\begin{prop}
    Let $X \subseteq \real^d$, $f,g: X \mapsto \real^m$, $a \in X$ such that $f,g$ are continuous at a,
    \begin{enumerate}[label=(\roman*)]
        \item $f+g: X \mapsto \real^m$ is continuous at $a$
        \item $cf: X \mapsto \real^m$ is continuous at $a$ for any $c \in \real$
        \item $\gamma f: X \mapsto \real^m$ is continuous at $a$ if $\gamma: X \mapsto \real$ is continuous at $a$. 
        \item If $m=1$, and $g(x) \neq 0$, $\forall x \in X$, then 
        \[\frac{f}{g}: X \mapsto \real\]
        is continuous at a.
    \end{enumerate}
\end{prop}
    \begin{proof}
        Follows from the algebra of limits.
    \end{proof}
    If $f: X \mapsto \real^m$, $f = (f_1, \ldots, f_m)$ where $f_i: X \mapsto \real$, then $f$ is continuous at $a$ if and only if $f_i$ is continuous at $a$ for all $i \in \{1, \ldots, m\}$.
    \begin{itemize}
        \item You are allowed to use the fact that sin, cos, exp, log are all continuous functions log are all continuous $\log: [0,\infty) \mapsto \real$ is continuous at all $a \in [0,\infty)$.
    \end{itemize}
    \section{Properties of Continuous Functions}
    $f: X \mapsto \real^m$ is \emph{(globally) continuous} if $f$ is continuous at $a$ for all $a \in X$.
    \begin{theorem}
        Let $K \subseteq \real^d$ be a compact set. If $f: K \mapsto \real^m$ is continuous, then $f(K)$ is compact. 
    \end{theorem}
    \begin{proof}
        Let $\infseq{a}$ be a sequence in $f(K)$. We need to show that $\exists$ a subsequence. We need to show that $\exists$ a subsequence $(a_{n_k})_{k=1}^\infty$ s.t $a_{n_k} \rightarrow b$ for some $b \in f(K)$. Since $a_n \in f(K)$, $\exists x_n \in K$ s.t $f(x_n) = a_n$ for all $n$. This gives a sequence $\infseq{x}$ in $K$. Since $K$ is compact, $\exists$ a subsequence $(x_{n_k})_{k=1}^\infty$ s.t $x_{n_k} \rightarrow x_0$. Since $f$ is continuous at $x_0$, $f(x_{n_k}) \rightarrow f(x_0)$. Therefore, $a_{n_k} \rightarrow f(x_0)$ and $(a_{n_k})_{n=1}^\infty$ is a subsequences of $(a_n)_{n=1}^\infty$. This is what we wanted.
    \end{proof} 
\begin{corollary}[Extreme Value Theorem]
    Let $K \subseteq \real^d$ be compact, $f: K \rightarrow \real$ a continuous function. Then $f$ is bounded and it attains its bounds, i.e $\exists x_{\min}, x_{\max} \in K$ such that $\forall x \in K$
    \[f(x_{\min}) \leq f(x) \leq f(x_{\max})\]
\end{corollary}
\begin{proof}
    From the previous result, $f(K)$ is compact in $\real$. By the Heine-Borel theorem, this means $f(k)$ is closed and bounded. Therefore, $f$ is bounded. Since $f(K)$ is closed, $\sup f(K) \in f(K)$ since there is a sequence $\infseq{y}$ in $f(K)$ such that $y_n \rightarrow \sup f(K)$. So $\exists x_{\max} \in K$ such that 
    \[f(x_{\max}) = \sup f(k)\]
    Therefore, 
    \[f(x) \leq f(x_{\max}) \text{  } \forall x \in K\]
    Similarily, $\exists x_{\min} \in K$ such that $f(x_{\min}) \leq f(x)$ $\forall x \in K$. 
\end{proof}

%----------------- Lecture 14
\chapter{Properties of Continuous Functions Continued}
\textbf{Recall:}
\begin{manualcorollary}{13.3.1}[Extreme Value Theorem]
    If $K \subseteq \real^d$ is compact and $f: K \rightarrow \real$ is continuous, then $\exists x_{\min}, x_{\max} \in K$ such that $f(x_{\min}) \leq f(x) \leq f(x_{\max})$ for all $x \in K$.
\end{manualcorollary} 
\begin{theorem}[Intermediate Value Theorem]
  Let $f: [a,b] \rightarrow \real$ be continuous, and let $y_0 \in \real$ be any number between $f(a)$ and $f(b)$. Then there exists a number $x_0$ between $a$ and $b$ such that $f(x_0) = y_0$.
\end{theorem}   
    \begin{proof}
        Without loss of generality, assume $f(a) \leq f(b)$. So $x_0 \in [f(a), f(b)]$. Let
        \[S \coloneqq \{x \in [a,b]: f(x) \leq x_0\}\]
        $S \subseteq [a,b]$, so $S$ is bounded. $a \in S$ since $f(a) \leq x_0$. Therefore $S \neq \emptyset$.
        So $\exists x_0 \coloneqq \sup S \in [a,b]$. We will show $f(x_0) = y_0$. We will consider the cases, where $f(x_0) = y_0$, $f(x_0) < y_0$, and $f(x_0) > y_0$. If $f(x_0) < y_0$, set $\epsilon \coloneqq y_0 - f(x_0)$. Since $f$ is continuous at $x_0$, $\exists \delta > 0$ s.t if $|x - x_0| < \delta$ and $x \in [a,b]$. Then $|f(x) - f(x_0)| < \epsilon$. Since $f(x_0) < y_0 \leq f(b)$ for $x_0 \neq b$. So we can find $x > x_0$ such that $x \in [a,b]$ and $|x - x_0| < \delta$.  Then $f(x) < f(x_0) + \epsilon = y_0$. So $x \in S$ by the definition of $S$, since $S = \{x: f(x) \leq y_0\}$. This is a contradiction since $x > x_0$, but $x_0$ is an upper bound for $S$.\\[2ex]
        If $f(x_0) > y_0$, set $\epsilon \coloneqq f(x_0) - y_0$. Since $f$ is continuous at $x_0$, $\exists \delta > 0$ such that if $x \in [a,b]$ and $|x - x_0| < \delta$, then $|f(x) - f(x_0)| < \epsilon$. So $f(x_0) > y_0 \leq f(a)$. So $x_0 > a$. We may assume that $x_0 - \delta > a$ since $\delta$ is abritrary and can be arbitrarly small. \\[1ex]
        Claim: $x_0 - \delta$ is an upper bound for $S$. Proof of claim, if $x > x - \delta$, then either $|x - x_0| < \delta$, in which case $f(x) > f(x_0)- \epsilon = y-0$, or $x > x_0$, in which case $x \not\in S$ since $x_0$ is an upper bound for $S$. Therefore, if $x > x_0 - \delta$, then $x \not\in S$. This proves the claim.\\[1ex]
        The claim contradits that $x_0$ is the least upper bound for $S$.  
    \end{proof}
    \begin{corollary}
        Let $f: [a,b] \rightarrow \real$ be continuous. Then $f([a,b]) = [c,d]$ for some $c \leq d$. 
    \end{corollary}
    \begin{proof}
        By extreme value theorem, $\exists x_{\min}, x_{\max}$ such that $f(x) \in [f(x_{\min}), f(x_{\max})]$ $\forall x \in [a,b]$. So we can set $c \coloneqq f(x_{\min})$ and $d \coloneqq f(x_{\max})$ and we have
        \[f([a,b]) \subseteq [c,d]\]
        for $y_0 \in [c,d] = [f(x_{\min}), f(x_{\max})]$. By the Intermediate Value Theorem, $\exists x_0$ between $x_{\min}$ nad $x_{\max}$ such that $f(x_0) = y_0$. Therefore
        \[[c,d] \subseteq f([a,b])\]
        Thus, 
        \[f([a,b]) = [c,d]\]
        \end{proof}

    \section{Inverses of Continuous Functions}
        Let $f: A \mapsto B$, $f$ is  
        \begin{itemize}
            \item \textbf{Injective} (or one-to-one) If $\forall x,y \in A$, if $f(x) = f(y)$, then $x = y$.
            \item \textbf{Surjective} (or onto) If $\forall y \in B$, $\exists x \in A$ such that 
            \[f(x) = y\]
            \item \textbf{Bijective} (or one-to-one and onto) If $f$ is both injective and surjective.
        \end{itemize}
        \begin{align*}
            \text{$f$ is bijective } &\iff \text{it is inveritble }\\
            &\iff \exists f^{-1}:B \mapsto A \text{ s.t } f \circ f^{-1} = id_B \text{ and } f^{-1} \circ f = id_A
        \end{align*}
    \begin{proof}
        ($\impliedby$) If $f$ is inveritble, then it is 
        \begin{itemize}
            \item \textbf{Injective} since if $f(x) = f(y)$ then $x = f^{-1}(f(x)) = f^{-1}(f(y)) = y$
            \item \textbf{Surjective} since given $y \in B$, $x \coloneqq f^{-1}(y)$ satisfies $f(x) = y$. 
        \end{itemize}
        ($\implies$) If $f$ is bijective, then for $y \in B$, $\exists x \in A$ such that $f(x) = y$. Also, this $x$ is unique since $f$ is injective. So we can define 
        \[f^{-1}(y) \coloneqq x\]
        and this is an inverse to $f$.
    \end{proof}
    Given $f: A \mapsto \real$ that is injective, we view $f$ as a function $A \mapsto f(A)$, and this way we force $f$ to be surjective. Thus $f^{-1}: f(A) \mapsto A$ exists.\\[2ex]
    \textbf{Question:} If $f$ is injective and continuous, must $f^{-1}$ be continuous? \textbf{No}. Consider the example with $f: [0,1) \cup [2, \infty]$
    \[f(x) \coloneqq \begin{cases}
        x_1 \text{ if } x < 1 \\
        x - 1 \text{ if } x \geq 2
    \end{cases}\]
    $f^{-1}: [0,\infty) \mapsto [0,1) \cup [2, \infty] \subseteq \real$, so 
    \[f^{-1}(y) = \begin{cases}
        y_1 \text{ if } y < 1 \\
        y + 1 \text{ if } y \geq 1
    \end{cases}\] 
    $f^{-1}$ is not continuous.
    \begin{definition}
        Let $A \subseteq \real$, $f: A \mapsto \real$. $f$ is 
        \begin{enumerate}[label=(\roman*)]
            \item \emph{increasing} if $\forall x,y \in A$, $x \leq y$ then $f(x) \leq f(y)$
            \item \emph{strictly increasing} if $\forall x,y \in A$, $x < y$ then $f(x) < f(y)$
            \item \emph{decreasing} if $\forall x,y \in A$, $x \leq y$ then $f(x) \geq f(y)$
            \item \emph{strictly decreasing} if $\forall x,y \in A$, $x < y$ then $f(x) > f(y)$
        \end{enumerate}
    \end{definition}
    \begin{lemma}
        Let $I \subseteq \real$ be an interval, $f: I \mapsto \real$ be an injective continuous function. Then $f$ is either strictly increasing or strictly decreasing. 
    \end{lemma}
    \begin{proof}
        It suffices to consider the case where $I$ is a closed, bounded interval. A general interval $I$ is a union of an increasing sequence of closed bounded intervals, so if $f$ is strictly increasing or strictly decreasing on each of these subintervals, then $f$ is strinctly increasing or decreasing on all of $I$. Consider the case $I = [a,b]$ with $(a < b)$. Without loss of generality, assume $f(a) < f(b)$, in this case we'll show that $f$ is strictly increasing. By contradiction, if $f$ is not strictly increasing, then there exists $x_1 < x_2$ such that 
        \[f(x_1) \geq f(x_2)\]
        We'll break this into 2 cases. 
        \begin{itemize}
            \item \textbf{Case 1:} $f(x) > f(b)$. In this case,
            \[f(a) < f(b) < f(x_1)\]
            and 
            \[f|_{[a,x_1]}\]
            is continuous. So by the Intermediate Value Theorem, $\exists z \in [a, x_1]$ such that $f(z) = f(b)$. But,
            \[z \leq x_1 < x_2 \leq b\]
            So $z \neq b$, which contradicts that $f$ is injective. 
            \item \textbf{Case 2:} $f(x_1) \leq f(b)$. In this case, $f(x_2) \leq f(x_1) \leq f(b)$ and $f|_{[x_1, b]}$ is continuous. So by the Intermediate Value Theorem, $\exists z \in [x_1, b]$ such that $f(z) = f(x_1)$. $z \geq x_2 > x_1$ so $z \neq x_1$. But, this contradicts that $f$ is injective.
        \end{itemize}
        In both cases, we get a contridction, so our assumption that $f$ is not strictly increasing is false.
    \end{proof}
    \chapter{}
    Let $I \subset \real$ be an interval, $f: I \mapsto \real$ be a continuous function. Then $f^{-1}: f(I) \mapsto \real$ is continuous. 
\end{document}